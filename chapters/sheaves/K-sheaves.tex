\documentclass[../../thesis.tex]{subfiles}

\begin{document}
Captured and made a prisoner of war by the Germans in 1940, the french professor and officer Jean Leray spent his five years in captivity inventing the theory of sheaves and spectral sequences.
Prior to being captured algebraic topology was merely a small interest and his real interests lied in using topological methods to prove the existence of solutions of certain differential equations, Leray feared that if the Germans learned of his competence as a ``mechanic'' (``mécanicien'' in his own words) they would force him to ``work for the German war machine, so he converted his minor interest to his major one, in fact to his essentially unique one, presented himself as a pure mathematician, and devoted himself mainly to algebraic topology.\cite{Borel}[p. 1-21].''
Originally, Leray defined sheaves as a functor from closed subspaces, and it was Cartan that reformulated the theory using open subspaces.\cite{H.Miller}
His first example was the sheaf assigning a space its $p$th cohomology group.
\TODO{Sign Cartan's 1950 seminar?}

We will closely follow Lurie \cite{HTT}.
Let $\Open(X)$ denote the partial order of open subsets of a topological space $X$.
\section{Sheaves on topological spaces}
\TODO{Define covering sieves.}
\begin{definition}[{\cite[Definition 7.3.3.1]{HTT}}]\label{sheaf_on_top}
    Let $X \in \cat{Top}$ and $\C$ an $\infty$-category.
    We define a $\C$-valued sheaf on $X$ to be a presheaf $\F: \Open(X)^{op} \to \C$ such that for every $U\subseteq X$ and every covering sieve $\mathscr{W} \subseteq \Open(X)_{/U}$, the diagram
    \[
        \Nerve(\mathscr{W})^{\triangleright} \hookrightarrow \Nerve(\Open(X)_{/U})^{\triangleright} \rightarrow \Nerve(\Open(X))\xrightarrow{\F}\C^{op}
    \]
    is a colimit.
\end{definition}
\prepp{This is equivalent to the following limit diagram:}
\[
    \Nerve((\mathscr{W})^{op})^{\triangleleft} \hookrightarrow \Nerve((\Open(X)_{/U})^{op})^{\triangleleft} \rightarrow \Nerve(\Open(X))^{op}\xrightarrow{\F}\C
\]
\TODO{Mention the ``normal definitions using covers and Cech nerves and briefly discuss relation to classical definition of sheaves in a $1$-category.}
We write $\Presh(X, \C)$ for the $\infty$-category $\mathrm{Fun}(\Open(X)^{op}, \C)$ of $\C$-valued presheaves on $X$ and $\Shv(X;\C)$ for the full subcategory of $\Presh(X;\C)$ spanned by the $\C$-valued sheaves on $X$.
Whenever we write $\Shv(X)$ without specifying the target category $\C$, we will always mean sheaves valued in spaces, i.e. $\Shv(X;\Spaces)$.
\TODO{Make sure $\Spaces$ is introduced in chapter on stable infcats.}
\section{Sheaves on locally compact spaces}
In this section we will show that for locally compact Hausdorff spaces there is an equivalence of $\infty$-categories between $\Shv(X;\C)$ and $\KShv(X;\C)$ where the latter denote so-called $\K$-sheaves and $\C$ is a presentable $\infty$-category with left exact filtered colimits.
These are sheaves defined on the collection of compact subsets instead of the opens.
Classically it is known that sheaves of sets on such spaces are determined by compact subsets as well as the opens.
\TODO{Expand on this with references.}
\begin{definition}
    For a locally compact Hausdorff space $X$, we write $\K(X)$ for its collection of compact subsets.
\end{definition}
\begin{definition}
    If $K, K' \subseteq X$, we write $K \Subset K'$ if there exists an open subset $U\subseteq X$ between $K$ and $K'$, i.e. $K \subseteq U \subseteq K'$.
\end{definition}
\begin{definition}
    If $K\subseteq X$ is compact, we write $\K_{K\Subset}(X)$ for the set $\{ K'\in \K(X) | K \Subset K' \}$ which gives a poset category $\K(X)$.
\end{definition}
\begin{definition}
    A presheaf $\F:\Nerve(\K(X))^{op}\to \C$ is a $\K$-sheaf if it satisfies the following:
    \begin{enumerate}
        \item $\F(\emptyset)$ is terminal.
        \item For every pair $K,K'\in \K(X)$, the diagram
              \[\begin{tikzcd}
                      {\F(K \cup K')} && {\F(K)} \\
                      \\
                      {\F(K')} && {\F(K\cap K')}
                      \arrow[from=1-3, to=3-3]
                      \arrow[from=3-1, to=3-3]
                      \arrow[from=1-1, to=3-1]
                      \arrow["\lrcorner"{anchor=center, pos=0.125}, draw=none, from=1-1, to=3-3]
                      \arrow[from=1-1, to=1-3]
                  \end{tikzcd}\]
              is a pullback in $\C$.
        \item For each $K\in \K(X)$, $\F(K)$ is a colimit of $\F|\Nerve(\K_{K\Subset}(X))^{op}$.
    \end{enumerate}
\end{definition}
\begin{definition}
    We denote the full subcategory of $\Presh(\Nerve(\K(X)); \C)$ spanned by the $\K$-sheaves by $\KShv(X;\C)$.
\end{definition}
\begin{lemma}[{\cite[][Lemma 7.3.4.8]{HTT}}]\label{7.3.4.8}
    Let $X$ be locally compact and Hausdorff, and let $\C$ be a presentable $\infty$-category with left exact filtered colimits.
    Let $\W$ be an open cover of $X$ and denote by $\K_{\W}(X)$ the compact subsets of $X$ that are contained in some element of $\W$.
    Any $\K$-sheaf $\F\in \KShv(X;\C)$ is a right Kan extension of $\F|\Nerve(\K_{\W}(X))^{op}$.
\end{lemma}
\begin{proof}
    \TODO{Proof is in HTT.}
\end{proof}
\begin{theorem}[{\cite[][Theorem 7.3.4.9]{HTT}}]\label{7.3.4.9}
    Let $X$ be locally compact and Hausdorff and $\C$ an $\infty$-category with left exact filtered colimits\footnote{Do I need to assumes existence of all small limits and colimits?}.
    Let $\F : \Nerve(\Open(X) \cup \K(X))^{op} \to \C$.
    The following conditions are equivalent:
    \begin{enumerate}[]
        \item The presheaf $\F_\K:= \F|\Nerve(\K(X))^{op}$ is a $\K$-sheaf, and $\F$ is a right Kan extension of $\F_\K$.
        \item The presheaf $\F_{\Open}:=\F|\Nerve(\Open(X))^{op}$ is a sheaf, and $\F$ is a left Kan extension of $\F_{\Open}$.
    \end{enumerate}
\end{theorem}
\prepp{I'm pretty sure we must regard the union $\Open(X) \cup \K(X)$ as a poset contained in the powerset of $X$ here for everything to make sense}
\begin{proof}
    \prepp{Proof is a bit long. Consider splitting it up.}
    \prepp{We regard $\K(X)$ as partially ordered by the $\Subset$ relation. (I think)}
    We start by assuming the first condition and want to show that $F$ is left Kan extended from $\Nerve({\Open}(X))^{op}$.
    By definition we want to show that
    \[
        \Nerve(\Open(X)_{/K}^{op})^{\triangleright} \hookrightarrow \Nerve((\Open(X)\cup \K(X))_{/K}^{op})^{\triangleright} \xrightarrow{c}\Nerve(\Open(X) \cup \K(X))^{op}\xrightarrow{\F}\C
    \]
    is a colimit diagram in $\C$.
    The assumption that $\F_{\K}$ is a $\K$-sheaf means that for each $K\in \K(X)$, $\F_{\K}(K)$ is a colimit of $\F|\Nerve(\K_{K\Subset}(X))^{op}$.
    We will "transfer" this colimit to the colimit we want by cofinal maps
    \[
        \Nerve((\Open(X)^{op})_{/K}) \xrightarrow{p} \Nerve((\Open(X)\cup \K(X))_{/K}^{op}) \xleftarrow{p'}\Nerve(\K_{K\Subset}(X))^{op}.
    \]
    Recall that by \ref{superlemma} checking cofinality reduces to checking weak contractibility of certain simplicial sets.
    For $p$ we must check $\Nerve((\Open(X))^{op}_{/K}) \times_{\Nerve((\Open(X)\cup \K(X))_{/K}^{op})} \Nerve((\Open(X)\cup \K(X))_{/K}^{op})_{K'/}$ is weakly contractible for every $K' \in \Nerve((\Open(X)\cup \K(X))_{/K}^{op})$.
    This is the simplicial set obtained by taking the nerve of the partially ordered set $\{U \in \Open(X) | K\subseteq U \subseteq K'\}$.
    By \cite[Lemma 5.3.1.20]{HTT} filtered $\infty$-categories are weakly contractible, and our partially ordered set is filtered as it is nonempty, stable under finite union, and taking nerves preserve the property of being filtered.
    The simplicial set $\Nerve(\{K'' \in \K(X) | K \Subset K'' \subseteq K'\})$\footnote{\TODO{This should probably be strict inclusion. Double check that.}} is weakly contractible by exactly the same argument, and hence $p$ and $p'$ are cofinal maps.
    By cofinality of $p$ and $p'$, the diagram
    \[
        \Nerve((\Open(X))^{op}_{/K})^{\triangleright} \hookrightarrow \Nerve((\Open(X)\cup \K(X))_{/K}^{op})^{\triangleright} \xrightarrow{c}\Nerve(\Open(X) \cup \K(X))^{op}\xrightarrow{\F}\C
    \]
    is a colimit diagram if and only if
    \[
        \Nerve((\K(X)_{\Subset K})^{op})^{\triangleright} \hookrightarrow \Nerve((\Open(X)\cup \K(X))_{/K}^{op})^{\triangleright} \xrightarrow{c}\Nerve(\Open(X) \cup \K(X))^{op}\xrightarrow{\F}\C
    \]
    is a colimit diagram, which it is by the assumption that $F_{\K}$ is a $\K$-sheaf.
    % \[\begin{tikzcd}
    %         {\Nerve(\Open(X)_{K\subseteq})^{op}} && {\Nerve(\Open(X)_{K\subseteq}\cup \K(X)_{K\Subset})^{op}} && {\Nerve(\K(X)_{K\Subset})^{op}} \\
    %         \\
    %         {\Nerve(\Open(X)_{K\subseteq}^{op})^{\triangleright}} && {\Nerve(\Open_{K\subseteq}(X)^{op}\cup \K(X)_{K\Subset}^{op})^{\triangleright}} && {\Nerve(\K(X)_{K\Subset}^{op})^{\triangleright}} \\
    %         \\
    %         && {\Nerve(\Open(X)\cup \K(X))^{op}} \\
    %         \\
    %         && \C
    %         \arrow["p", from=1-1, to=1-3]
    %         \arrow["{p'}"', from=1-5, to=1-3]
    %         \arrow[from=1-1, to=3-1]
    %         \arrow[from=3-1, to=3-3]
    %         \arrow[from=3-5, to=3-3]
    %         \arrow[from=1-3, to=3-3]
    %         \arrow[from=3-3, to=5-3]
    %         \arrow["\F", from=5-3, to=7-3]
    %         \arrow["{\psi'}"', from=3-5, to=7-3]
    %         \arrow["\psi", from=3-1, to=7-3]
    %         \arrow[from=1-5, to=3-5]
    %     \end{tikzcd}\]


    We now show $\F_{\Open}$ is a sheaf.
    By definition \ref{sheaf_on_top} we must show that for every $U\in \Open(X)$ and every covering sieve $\mathscr{W}$ covering $U$,
    \[
        \Nerve(\W)^{\triangleright} \hookrightarrow \Nerve(\Open(X)_{/U})^{\triangleright} \rightarrow \Nerve(\Open(X))\xrightarrow{\F}\C^{op}
    \]
    is a colimit diagram, or equivalently that
    \[
        \Nerve(\W)^{op,\triangleleft} \hookrightarrow \Nerve(\Open(X)_{/U})^{op,\triangleleft} \rightarrow \Nerve(\Open(X))^{op}\xrightarrow{\F}\C
    \]
    is a limit diagram.
    We will once again use cofinality by observing that \ref{superlemma} implies cofinality of the inclusion
    \[
        \Nerve(\W) \subseteq \Nerve(\W \cup \K_{\W}(X))
    \]
    where $\K_{\W}(X)$ is the set $\{K\in \K(X) | (\exists W \in \W)[K \subseteq W] \}$, so it is enough to show the limit starting from $ \Nerve(\W \cup \K_{\W}(X))^{op}$.
    \TODO{Write out the details of why this is contractible and the inclusion is cofinal.}
    Recall that this is equivalent to showing that $\F|_{\Nerve(\W \cup \K_{\W}(X))^{op,\triangleleft}}$ is right Kan extended from $\F|_{\Nerve(\W \cup \K_{\W}(X))^{op}}$.
    \TODO{Write out the result in the Kan chapter.}
    By the assumption that $\F$ is a right Kan extension of $\F_{\K}$ and the observation that
    \[
        \F(U) = \lim_{K\in \K(X)^{op}_{U/}}\F(K) = \lim_{K\in \K_{\W}(X)^{op}_{U/}}\F(K) %Do we need the following?= K_*^{\prime}\F(U)
    \]
    we see that $\F|_{(\W \cup \K_{\W})^{op}}$ is a right Kan extension of $\F|_{(K_{\W}(X))^{op}}$.
    Hence, it suffices to prove that $\F|_{(\W \cup \K_{\W}(X)\cup \{U\})^{op}}$ is right Kan extended from $\K_{\W}(X)^{op}$.
    Outside of $U$ this is clear from the fact that $F|_{(\W \cup \K_{\W}^{op})}$ is right Kan extended from $\K_{\W}(X)^{op}$.
    This means we only need to prove $\F|_{(\K_{\W}(X)\cup \{U\})^{op}}$ is a right Kan extension of $\F|_{\K_{\W}(X)^{op}}$.
    %\prepp{I'm honestly not entirely sure why we suddenly drop the $\cup W$ part.}
    %\prepp{$(\W \cup \K_{\W}(X) \cup \{U\})^{op} = (\W \cup \K_{\W}(X))^{op, \triangleleft}$}
    Observe that by assumption
    \[
        \F(U)  = \lim_{K\in \K(X)_{/U}^{op}} \F(K) = \lim_{K\in (\K(X)_{/U} \cup \{U\})^{op}} \F(K)
    \]
    so $\F|_{(\K(X)_{/U} \cup \{U\})^{op}}$ is right Kan extended from $\K(X)_{/U}^{op}$.
    Lemma \ref{7.3.4.8} tells us that $\F|_{(\K(X)_{/U})^{op}}$ is a right Kan extension of $\F|_{\K_{\W}(X)^{op}}$.
    We have ${\Nerve(\K_{\W}(X))^{op} \subseteq\Nerve(\K(X)_{/U})^{op} \subseteq\Nerve(\K(X)_{/U} \cup \{U\})^{op}}$, with Kan extensions as in proposition \ref{4.3.2.8}, so we get that ${\F|_{\Nerve(\K(X)_{/U} \cup \{U\})^{op}}}$ is right Kan extended from ${\Nerve(\K_{\W}(X))^{op}}$.
    To summarize, we have the following square of inclusions
    \[\begin{tikzcd}
            {\K_{\W}(X)^{op}} && {(\K_{\W}(X)\cup \{U\})^{op}} \\
            \\
            {\K(X)_{/U}^{op}} && {(\K(X)_{/U} \cup \{U\})^{op}}
            \arrow["{b}", hook, from=1-3, to=3-3]
            \arrow["{j}", hook, from=3-1, to=3-3]
            \arrow["{a}", hook, from=1-1, to=3-1]
            %\arrow[from=1-1, to=3-3]
            \arrow["{i}" ,hook, from=1-1, to=1-3]
        \end{tikzcd}\]
    %where $\F|(\K(X)_{/U} \cup \{U\})^{op} = j_*(\F|\K(X)_{/U})$ and $\F|(\K(X)_{/U})^{op} = j_*(\F|\K_{\W}(X)^{op})$. By applying proposition \ref{4.3.2.8} we see that ${\F|\Nerve(\K(X)_{/U} \cup \{U\})^{op}}= (j \circ a)_*(\F|\K_{\W}(X)^{op})$.
    where $\F|_{(\K(X)_{/U} \cup \{ U \})^{op}} \simeq b_*(\F|_{(\K_{\W} \cup \{U\})^{op}})$ and $\F|_{(\K(X)_{/U} \cup \{U\})^{op}} \simeq (j\circ a)_*(\F|_{\K_{\W}(X)^{op}})$.
    We want to show $\F|_{(\K_{\W}(X) \cup \{U\})^{op}} \simeq i_*(\F|_{\K_{\W}(X)^{op}})$.
    Since $b$ is fully faithful (it is the inclusion of a full subcategory), we know $b^*b_* \simeq \id$, so we get
    \begin{align}
        \F|_{(\K_\W(X)\cup \{U\})^{op}} & \simeq b^*b_*(\F|_{(\K_\W(X)\cup \{U\})^{op}}) \\
                                        & \simeq b^*(j\circ a)_* (\F|_{\K_\W(X)^{op}})   \\
                                        & \simeq b^*(b \circ i)_*(\F|_{\K_\W(X)^{op}})   \\
                                        & \simeq b^*b_*i_*(\F|_{\K_\W(X)^{op}})          \\
                                        & \simeq i_*(\F|_{\K_\W(X)^{op}})
    \end{align}
    %\prepp{This is probably obvious or easier to show than what I am about to do, so please let me know.}
    % We use the commutativity of the square and the adjoint relationship between Kan extensions and ``pullbacks'' and get:
    % \begin{align}
    %     \C^{(\K_{\W}(X)\cup \{U\})^{op}}(b^*(\blank),\F|_{(\K_{\W}(X)\cup \{U\})^{op}})\simeq & \C^{(\K(X)_{/U} \cup \{U\})^{op}}(\blank, \F)                             \\       \simeq                                                                             & \C^{(\K(X)_{/U} \cup \{U\})^{op}}(\blank,a_*(\F|_{{\K\W}(X)^{op}})) \\
    %     \simeq                                                                                & \C^{\K_{\W}(X)^{op}}(a^*(\blank), \F|_{\K_{\W}(X)^{op}})                  \\
    %     \simeq                                                                                & \C^{\K_{\W}(X)^{op}}((b\circ i)^*(\blank), \F|_{\K_{\W}(X)^{op}})         \\        \simeq                                                                             & \C^{\K_{\W}(X)^{op}}(i^* \circ b^*(\blank), \F|_{\K_{\W}(X)^{op}})            \\
    %     \simeq                                                                                & \C^{(\K_{\W}(X)\cup \{U\})^{op}}(b^*(\blank), i_*(\F|_{\K_{\W}(X)^{op}}))
    % \end{align}
    % and Yoneda's lemma tells us that
    % \[\C^{(\K_{\W}(X)\cup \{U\})^{op}}(b^*(\blank),\F|_{(\K_{\W}(X)\cup \{U\})^{op}}) \simeq \C^{(\K_{\W}(X)\cup \{U\})^{op}}(b^*(\blank), i_*(\F|_{\K_{\W}(X)^{op}}))\]
    % \[\Rightarrow \F|_{(\K_{\W}(X)\cup \{U\})^{op}} \simeq i_*(\F|_{\K_{\W}(X)^{op}}).\]
    In conclusion we have shown that $\F_{U}$ is a sheaf, so $1.$ implies $2.$.


    For the other direction we assume $2.$ and want to show $\F_{\K}$ is a $\K$-sheaf. By definition we need to show three things:
    Firstly, observe that $\F_{\K}(\emptyset) = \F_{U}(\emptyset)$ and since $\F_{U}$ is a sheaf $\F_{\K}(\emptyset)$ is terminal.
    Secondly, we need the following diagram to be a pullback in $\C$ for any $K,K' \in \K(X)$.
    \begin{equation}\label{K-sheaf-PB}
        \begin{tikzcd}
            {\F(K \cup K')} && {\F(K)} \\
            \\
            {\F(K')} && {\F(K\cap K')}
            \arrow[from=1-3, to=3-3]
            \arrow[from=3-1, to=3-3]
            \arrow[from=1-1, to=3-1]
            %\arrow["\lrcorner"{anchor=center, pos=0.125}, draw=none, from=1-1, to=3-3]
            \arrow[from=1-1, to=1-3]
        \end{tikzcd}
    \end{equation}
    We will do this by using that $\F_{U}$ is a sheaf.
    Let $P = \{(U,U') | K \subseteq U, K' \subseteq U'\}$ and $\sigma: \Delta^1\times\Delta^1 \to \C$ denote diagram \ref{K-sheaf-PB}.
    Now $\F$ induces a map $\sigma_P: \Nerve(P^{op})^{\triangleright} \to \C^{\Delta^1\times\Delta^1}$ taking each $(U, U')$ to
    \[
        \begin{tikzcd}
            {\F(U \cup U')} && {\F(U)} \\
            \\
            {\F(U')} && {\F(U\cap U')}
            \arrow[from=1-3, to=3-3]
            \arrow[from=3-1, to=3-3]
            \arrow[from=1-1, to=3-1]
            \arrow["\lrcorner"{anchor=center, pos=0.125}, draw=none, from=1-1, to=3-3]
            \arrow[from=1-1, to=1-3]
        \end{tikzcd}
    \]
    and the cone point is sent to $\sigma$. This is a pullback by the fact that $\F_{U}$ is a sheaf.
    Evaluating $\sigma_P$ in each of the four vertices of $\Delta^1 \times \Delta^1$ we get four maps $\Nerve(P^{op})^{\triangleright} \to \C$.
    We now check that evaluating in the final vertex yields a colimit diagram.
    %Let $Q = \{ U\in \Open(X) | K\cap K' \subseteq U \}$, and observe that this is the fiber product $\Open(X)_{/(K\cap K')}$.
    By assumption $\F$ is a left Kan extension of $\F_{U}$ which by definition means that the following is a colimit diagram:
    \[
        \Nerve((\Open(X)_{/(K\cap K')})^{op})^{\triangleright} \hookrightarrow \Nerve((\Open(X) \cup \K(X))^{op}_{/(K\cap K')})^{\triangleright} \xrightarrow{c} \Nerve(\Open(X) \cup \K(X))^{op} \xrightarrow{\F}\C
    \]
    Observe that for every $U'' \in \Open(X)_{/(K\cap K')}$, the set
    $P_{U''} = \{(U,U') \in P | U\cap U' \subseteq U''\}$ is nonempty and stable under finite intersections, which implies it is filtered and hence its nerve is contractible.
    \TODO{Reference this result.}
    By \ref{superlemma} this implies $\Nerve(P^{op}) \to \Nerve((\Open(X)_{/(K\cap K')})^{op})$ is cofinal and we get a colimit diagram
    \[
        \Nerve(P^{op})^{\triangleright} \to\Nerve((\Open(X)_{/(K\cap K')})^{op})^{\triangleright} \hookrightarrow \Nerve((\Open(X) \cup \K(X))_{/(K\cap K')}^{op})^{\triangleright} \xrightarrow{c} \Nerve(\Open(X) \cup \K(X))^{op} \xrightarrow{\F}\C.
    \]
    We can show that evaluating the three other vertices also yields colimit diagrams by similar arguments.
    \prepp{I should maybe do this for all four vertices, or at least check how different the argument is for another vertex, but it looks pretty similar.}
    Since $\sigma_P$ yields a colimit diagram when evaluated in each of the four vertices of $\Delta^1 \times \Delta^1$, we conclude that $\sigma_P$ is itself a colimit diagram by \cite[Proposition 5.1.2.2]{HTT}.
    Observe now that $\sigma_P$ is a filtered colimit in $\C$ and hence it is left exact.
    This concludes the argument that \ref{K-sheaf-PB} is a pullback.
    Finally, we need to show that for each $K \in \K(X)$, $\F_{\K}$ is a colimit of $\F_{\K}|_{\Nerve(\K_{K\Subset}(X))^{op}}$.
    We do this by showing
    \[
        \Nerve(\K_{K\Subset}(X)^{op})^{\triangleright} \to \Nerve(\K(X) \cup \Open(X))^{op} \xrightarrow{\F} \C
    \]
    is a colimit diagram.
    We use Proposition \ref{4.3.2.8} to show that
    $\F|_{\Nerve(\Open(X) \cup \K_{K\Subset}(X))^{op}}$ and ${\F|_{\Nerve(\Open(X) \cup \K_{K\Subset}(X))^{op} \cup \{K\}}}$ are left Kan extensions of $\F|_{\Nerve(\Open(X))^{op}}$ which again implies $\F|_{\Nerve(\Open(X) \cup \K_{K\Subset}(X))^{op} \cup \{K\}}$ is a left Kan extension of $\F|_{\Nerve(\Open(X) \cup \K_{K\Subset}(X))^{op}}$.
    Now observe that
    \[
        \Nerve(\K(X) \cup \Open(X))^{op,\triangleright}_{/K} = \Nerve(\K_{K\Subset}(X) \cup \Open(X)_{/K})^{op,\triangleright},
    \]
    so in particular
    \[
        \Nerve(\K(X) \cup \Open(X))^{op,\triangleright}_{/K} \to \Nerve(\K(X) \cup \Open(X))^{op} \xrightarrow{\F} \C
    \]
    is a colimit diagram, and the statement is reduced to showing that $\Nerve(\K_{K\Subset}(X)) \subseteq \Nerve(\K(X) \cup \Open(X))_{/K}^{op}$ is cofinal.
    Let $Y\in \Nerve(\K(X) \cup \Open(X))_{/K}$ and let $R$ be the partially ordered set $\{K' \in \K(X) | K \Subset K ' \subseteq Y\}$.
    Since $R$ is nonempty and stable under intersections, $R^{op}$ is filtered and hence $\Nerve(R)$ is weakly contractible.
    By Lemma \ref{superlemma} the inclusion is cofinal and we have shown that $\F_\K$ is a $\K$-sheaf.

    We will show $\F$ is a right Kan extension of $\F_\K$ in a similar manner to how we showed $\F$ is a left Kan extension of $\F_{\Open}$ in the start of the proof, but we will consider the partial order on $\Open(X)$ given by writing $V\Subset U$ whenever $V\in \Open(X)$ and its closure $\overline{V}$ is compact and contained in $U$.
    Writing $\Open(X)_{U/}$ for the set $\{V\in \Open(X) | V\Subset U\}$, we need to show that
    \[
        \Nerve(\K(X)_{U/}^{op})^{\triangleleft} \hookrightarrow \Nerve(\Open(X) \cup \K(X))_{U/}^{op} \xrightarrow{c}\Nerve(\Open(X) \cup \K(X))^{op} \xrightarrow{\F} \C
    \]
    is a colimit diagram.
    As earlier we do this by finding cofinal inclusions
    \[
        \Nerve(\K(X)_{U/}^{op}) \xrightarrow{f}\Nerve(\Open(X) \cup \K(X))_{U/}^{op} \xleftarrow{f'} \Nerve(\K(X)_{/U})^{op}.
    \]
    By Lemma \ref{superlemma} $f$ and $f'$ are cofinal inclusions if for any $Y\in (\Open(X) \cup \K(X))_{U/}$ the partially ordered sets
    \[
        \{ V \in \Open(X) |Y \subseteq V \Subset U \}
    \]
    and
    \[
        \{ K \in \K(X) |Y \subseteq K \subseteq U \}
    \]
    have weakly contractible nerves, which they have by the usual argument; they are nonempty and stable under unions, hence filtered.
    \TODO{Give this argument a name and discuss it in an earlier section.}
    Since $\Open(X)_{U/}$ is a sieve covering $U$ and $\F_U$ is a sheaf,
    \[
        \Nerve(\Open(X)_{U/})^{op} \to \Nerve(\Open(X)_{U/})^{op, \triangleleft} \to \C
    \]
    is a colimit diagram and this completes the proof that $\F$ is a right Kan extension of $\F_\K$.
\end{proof}
\begin{corollary}\label{7.3.4.10}
    Let $X$ be a locally compact Hausdorff space and $\C$ an $\infty$-category with left exact filtered colimits, then $\Shv(X;\C) \simeq \KShv(X;\C)$ is an equivalence of $\infty$-categories.
\end{corollary}
\begin{proof}
    Let $\Shv_{\K\Open}(X;C)$ be the full subcategory of $\Fun(\Nerve(\K(X)\cup \Open(X))^{op},\C)$ spanned by those presheaves satisfying the equivalent conditions of theorem \ref{7.3.4.9}. We get restrictions
    \[
        \Shv(X;\C) \leftarrow \Shv_{\K\Open}(X;C) \to \KShv(X;C),
    \]
    and these are equivalences of $\infty$-categories because inclusions of full subcategories are fully faithful and Kan extensions along fully faithful functors give isomorphisms.

    ($K$ fully faithful means $G \cong Lan_K(G)K$.)
\end{proof}
\end{document}