\documentclass[../../thesis.tex]{subfiles}

\begin{document}
Captured and made a prisoner of war by the Germans in 1940, the french professor and officer Jean Leray spent his five years in captivity inventing the theory of sheaves and spectral sequences.
Prior to being captured algebraic topology was merely a small interest and his real interests lied in using topological methods to prove the existence of solutions of certain differential equations, Leray feared that if the Germans learned of his competence as a ``mechanic'' (``mécanicien'' in his own words) they would force him to ``work for the German war machine, so he converted his minor interest to his major one, in fact to his essentially unique one, presented himself as a pure mathematician, and devoted himself mainly to algebraic topology.\cite{Borel}[p. 1-21].''
Originally, Leray defined sheaves as a functor from closed subspaces, and it was Cartan that reformulated the theory using open subspaces.\cite{H.Miller}
His first example was the sheaf assigning a space its $p$th cohomology group.
\TODO{Cite Cartan's 1950 seminar?}

We will closely follow Lurie \cite{HTT}.
Let $\Open(X)$ denote the partial order of open subsets of a topological space $X$.
\section{Sheaves on topological spaces}
\TODO{Define covering sieves.}
\begin{definition}[{\cite[Definition 7.3.3.1]{HTT}}]\label{sheaf_on_top}
    Let $X \in \cat{Top}$ and $\C$ an $\infty$-category.
    We define a $\C$-valued sheaf on $X$ to be a presheaf $\F: \Open(X)^{op} \to \C$ such that for every $U\subseteq X$ and every covering sieve $\mathscr{W} \subseteq \Open(X)_{/U}$, the diagram
    \[
        \Nerve(\mathscr{W})^{\triangleright} \hookrightarrow \Nerve(\Open(X)_{/U})^{\triangleright} \rightarrow \Nerve(\Open(X))\xrightarrow{\F}\C^{op}
    \]
    is a colimit.
\end{definition}
\begin{remark}
    We will often utilize the fact that this is equivalent to the following limit diagram:
    \[
        \Nerve((\mathscr{W})^{op})^{\triangleleft} \hookrightarrow \Nerve((\Open(X)_{/U})^{op})^{\triangleleft} \rightarrow \Nerve(\Open(X))^{op}\xrightarrow{\F}\C
    \]
\end{remark}

\TODO{Mention the ``normal definitions using covers and Cech nerves and briefly discuss relation to classical definition of sheaves in a $1$-category.}
We write $\Presh(X, \C)$ for the $\infty$-category $\mathrm{Fun}(\Open(X)^{op}, \C)$ of $\C$-valued presheaves on $X$ and $\Shv(X;\C)$ for the full subcategory of $\Presh(X;\C)$ spanned by the $\C$-valued sheaves on $X$.
Whenever we write $\Shv(X)$ without specifying the target category $\C$, we will always mean sheaves valued in spaces, i.e. $\Shv(X;\Spaces)$.
\TODO{Make sure $\Spaces$ is introduced in chapter on stable infcats.}
\section{Sheaves on locally compact spaces}
In this section we will show that for locally compact Hausdorff spaces there is an equivalence of $\infty$-categories between $\Shv(X;\C)$ and $\KShv(X;\C)$ where the latter denote so-called $\K$-sheaves and $\C$ is a presentable $\infty$-category with left exact filtered colimits.
These are sheaves defined on the collection of compact subsets instead of the opens.
Classically it is known that sheaves of sets on such spaces are determined by compact subsets as well as the opens.
\TODO{Expand on this with references.}
\begin{definition}
    For a locally compact Hausdorff space $X$, we write $\K(X)$ for its collection of compact subsets.
\end{definition}
\begin{definition}
    If $K, K' \subseteq X$, we write $K \Subset K'$ if there exists an open subset $U\subseteq X$ between $K$ and $K'$, i.e. $K \subseteq U \subseteq K'$.
\end{definition}
\begin{definition}
    If $K\subseteq X$ is compact, we write $\K_{K\Subset}(X)$ for the set $\{ K'\in \K(X) | K \Subset K' \}$ which gives a poset category $\K(X)$.
\end{definition}
\begin{definition}
    A presheaf $\F:\Nerve(\K(X))^{op}\to \C$ is a $\K$-sheaf if it satisfies the following:
    \begin{enumerate}
        \item $\F(\emptyset)$ is terminal.
        \item For every pair $K,K'\in \K(X)$, the diagram
              \[\begin{tikzcd}
                      {\F(K \cup K')} && {\F(K)} \\
                      \\
                      {\F(K')} && {\F(K\cap K')}
                      \arrow[from=1-3, to=3-3]
                      \arrow[from=3-1, to=3-3]
                      \arrow[from=1-1, to=3-1]
                      \arrow["\lrcorner"{anchor=center, pos=0.125}, draw=none, from=1-1, to=3-3]
                      \arrow[from=1-1, to=1-3]
                  \end{tikzcd}\]
              is a pullback in $\C$.
        \item For each $K\in \K(X)$, $\F(K)$ is a colimit of $\F|\Nerve(\K_{K\Subset}(X))^{op}$.
    \end{enumerate}
\end{definition}
\begin{definition}
    We denote the full subcategory of $\Presh(\Nerve(\K(X)); \C)$ spanned by the $\K$-sheaves by $\KShv(X;\C)$.
\end{definition}
\begin{lemma}[{\cite[][Lemma 7.3.4.8]{HTT}}]\label{7.3.4.8}
    Let $X$ be locally compact and Hausdorff, and let $\C$ be a presentable $\infty$-category with left exact filtered colimits.
    Let $\W$ be an open cover of $X$ and denote by $\K_{\W}(X)$ the compact subsets of $X$ that are contained in some element of $\W$.
    Any $\K$-sheaf $\F\in \KShv(X;\C)$ is a right Kan extension of $\F|\Nerve(\K_{\W}(X))^{op}$.
\end{lemma}
\begin{proof}
    \TODO{Proof is in HTT.}
\end{proof}
\end{document}