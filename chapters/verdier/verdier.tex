\documentclass[../../thesis.tex]{subfiles}
\begin{document}
In this chapter we will prove an $\infty$-categorical analog of the classical Verdier duality theorem.
While classical Verdier duality is usually phrased in the form of an adjunction, we will prove an equivalence of $\infty$-categories instead.
More precisely we will show that for a locally compact Hausdorff topological space $X$ and stable $\infty$-category $\C$ there is an equivalence between the $\infty$-category of $\C$-valued sheaves on $X$ and the $\infty$-category of $\C$-valued cosheaves on $X$.
The main result of the last chapter (Theorem \ref{7.3.4.9}) will be crucial as it let's us show the equivalence on the level of $\K$-sheaves instead.
\section{Classical Verdier Duality}
Classical Verdier duality can be seen as a generalization of Poincaré duality, replacing the pairing on cohomology with a pairing in the derived category.
It was introduced in 1965 by Jean Louis Verdier \cite{Verdier95}.
Grothendieck had, a few years earlier, introduced a Poincaré duality in étale cohomology for schemes in algebraic geometry and Verdier duality serves as an analog of this duality for the theory of locally compact topological spaces.
Verdier duality applies to morphisms of locally compact spaces and reduces to classical Poincaré duality when considering the projection of a manifold to the point.
Recall that for a map $X \to Y$ between topological spaces and a sheaf $\F$ on $X$ we can define a so-called direct image sheaf $f_*\F$ on $Y$ by considering preimages of $f$:
\[
    f_*\F(U) : = \F(f^{-1}(U)) \text{ for any }  U \in \Open(Y)
\]
Furthermore, viewing $f_*$ as a functor $\Shv(X;\C) \to \Shv(Y; \C)$, there is a left adjoint $f^*$ taking a sheaf $\G$ on $Y$ to the so-called inverse image sheaf on $X$.
Explicitly, this sheaf is given by the sheafification of the presheaf given by the formula
\[
    V \mapsto \colim_{U \supseteq f(V)}\G(U),
\]
where $V$ is any open subset of $X$ and the colimit is taken over all open subsets $U$ of $Y$ containing $f(V)$.
Observe that considering the case when $Y$ is just the point, the projection map $f$ induces a functor $f_* \colon \Shv(X;\C) \to \Shv(*; \C) \simeq \C$; this is the global sections functor $\Gamma(X; \blank)$.
As sheaf cohomology is defined as the right derived functor of the global sections functor, we can write
\[
    R^i f_* (\F) \cong R^i\Gamma(X;\blank)(\F) = H^i(X; \F)
\]
and this motivates the perspective of seeing the derived direct image as a kind of relative version of cohomology.
Next, one constructs a so-called proper (or exceptional) direct image functor $f_!$ by considering the compactly supported subfunctor of $f_*$:
\[
    \Gamma(U; f_!\F) : = \{ s \in \Gamma(U; f_*\F) | s \text{ has compact support}\}.
\]
If we once again consider the special case where $Y = *$, one recovers compactly supported cohomology:
\[
    R^i f_! (\F) \cong H_c^i(X; \F)
\]
The natural follow-up question to seeing this construction is: what about proper inverse image?
In general $Rf_!$ does not need to admit a right adjoint, but for locally compact Hausdorff spaces it exists and we denote it $f^!$.
Together with the tensor product and the internal hom, the four functors $f_*,f^*,f_!$ and $f^!$ produce a six functor formalism.
It is this final adjunction we call Verdier Duality:
\begin{theorem}
    Let $f \colon X \to Y$ be a continuous map of locally compact Hausdorff spaces such that $f_!$ has finite cohomological dimension\footnote{This means that there is some bound $d$ for which all cohomology groups $H^r_c(f^{-1}(y);\mathrm{A})$ above $d$ vanish for all $y\in Y$. This holds for instance if all the fibres $f^{-1}(y)$ are at most $d$-dimensional CW-complexes.}.
    Then the derived functor of the proper direct image has a right adjoint $f^!$ in the derived category of sheaves:
    \[
        R\Hom_{D(\Shv(Y;\C))}(Rf_!\F,\G) \cong R\Hom_{D(\Shv(X;\C))}(\F,f^!\G),\footnote{Here we really mean complexes of sheaves.}
    \]
    where $\F$ and $\G$ are sheaves of for instance $\mathrm{A}$-modules.
\end{theorem}
Again, taking $f$ to be the projection $X \to *$ yields fruitful special cases.
We consider the simpler case when $X$ is a compact and orientable $n$-dimensional manifold and consider the constant sheaf $k_X$ on $X$ for some field $k$.
Verdier Duality now gives an equivalence
\[
    \Hom(Rf_!(k_X) , k) \cong \Hom(k_X , f^!k).
\]
Using the fact that we can find an injective resolution $I^{\bullet}$ of $k_X$ and some homological algebra, one can show that maps $\Hom(Rf_!(k_X) , k)$ are further equivalent to the zeroth cohomology group with compact support
\[
    H^0(\Hom^{\bullet}(\Gamma_c(X;I^{\bullet}), k)) = H_c^0(X;k_X)^{\vee},
\]
where $\vee$ denotes the dual vector space.
Additionally, it can be shown that $f^!k = k_X[n]$, which is something called the dualizing complex for a manifold.
This means that studying $\Hom^{\bullet}(k_X ,f^!k)$ reduces to studying $\Hom(k_X , k_X[n])$ which is isomorphic to the $n$-th cohomology of $\Hom^{\bullet}(k_x, k_x)$, so we have an isomorphism
\[
    H^0_c(X; k_X)^{\vee} \cong H^n(X;k_X).
\]
Now, shifting the complexes $i$ degrees and repeating the above arguments recovers classical Poincaré duality:
\[
    H^i_c(X; k_X)^{\vee} \cong H^{n-i}(X;k_X).
\]
For a more detailed and precise account of the above arguments, see Iversen's book \cite{Iversen} which serves as a good introduction to sheaves and sheaf cohomology as a whole and for a slightly more amateurish account, see \cite{BACH}.
% \begin{theorem}
%     Let $f : X \to Y$ be a proper map between manifolds.
%     Then the derived functor $\mathbf{R}f_* : \mathbf{D}^+(X,k) \to \mathbf{D}^+(Y,k)$ admits a right adjoint $f^!:\mathbf{D}^+(Y,k) \to \mathbf{D}^+(X,k)$.
% \end{theorem}

\section{Verdier Duality in stable $\infty$-categories}
The goal of this chapter is to prove the following theorem:
\begin{theorem}[Verdier Duality {\cite[Theorem 5.5.5.1]{HA}}]\label{VerdierDuality}
    Let $X$ be a locally compact Hausdorff space and $\C$ be a stable $\infty$-category with small limits and colimits.
    Then we have an equivalence of $\infty$-categories
    \[
        \mathbb{D} \colon \Shv(X;\C)^{op} \simeq \Shv(X;\C^{op}).
    \]
\end{theorem}
Let $k$ be a field and $\mathrm{Ch}_k$ the category of chain complexes of $k$-vector spaces.
We can define the derived $\infty$-category $D(\mathrm{Ch}_k)$ for $k$ by inverting all quasi-isomorphisms in $\mathrm{Ch}_k$.
Vector space duality gives a limit preserving functor $D(\mathrm{Ch}_k^{op}) \to D(\mathrm{Ch}_k)$ which induces a functor
\[
    \Shv(X;D(\mathrm{Ch}_k)^{op}) \to \Shv(X;D(\mathrm{Ch}_k))
\]
for any locally compact Hausdorff space.
Composing with the equivalence of Theorem \ref{VerdierDuality} yields a contravariant functor from $\Shv(X;D(\mathrm{Ch}_k))$ to itself:
\[
    \mathbb{D}'\colon \Shv(X;D(\mathrm{Ch}_k))^{op} \to \Shv(X;D(\mathrm{Ch}_k))
\]
and it is this functor that is usually called Verdier Duality. As we have composed the equivalence $\mathbb{D}$ with vector space duality, $\mathbb{D}'$ is not necessarily an equivalence of $\infty$-categories unless certain finiteness conditions are imposed.


We will be using the theory of $\K$-sheaves set up in the previous chapter to prove the theorem.
By corollary \ref{7.3.4.10} we can rewrite theorem \ref{VerdierDuality} in terms of $\K$-sheaves instead:
\begin{theorem}\label{KVerdierDuality}
    Let $X$ be a locally compact Hausdorff space and $\C$ be a stable $\infty$-category with small limits and colimits.
    Then we have an equivalence of $\infty$-categories:
    \[
        \mathbb{D}_{\K} \colon \KShv(X;\C)^{op} \simeq \KShv(X;\C^{op}).
    \]
\end{theorem}
\begin{definition}[{\cite[Notation 5.5.5.5]{HA}}]\label{5.5.5.5}
    Let $X$ be a locally compact Hausdorff space.
    We define a partially ordered set $M$ as follows:
    \begin{enumerate}
        \item The objects of $M$ are pairs $(i,S)$ where $0 \leq i \leq 2$ and $S \subseteq X$ such that $i=0$ implies $S$ is compact and $i=2$ implies $X-S$ is compact.
        \item We have $(i,S) \leq (j,T)$ if either $i\leq j$ and $S\subseteq T$, or $i=0$ and $j=2$.
    \end{enumerate}
\end{definition}
\begin{remark}[{\cite[Remark 5.5.5.6]{HA}}]
    Observe that projecting $(i,S) \to i$ gives a map $\varphi: M \to [2]$ of partially ordered sets.
    For $0 \leq i \leq 2$ denote the fiber $\varphi^{-1}\{i\}$ by $M_i$.
    Also, observe that $M_0 \cong \K(X), M_2\cong \K(X)^{op}$ and $M_1$ is isomorphic to the powerset poset of $X$.
\end{remark}
\begin{definition}
    Let $M'$ denote the partially ordered sets of pairs $(i,S)$, where $0 \leq i \leq 2$ and $S \subseteq X$ such that $i=0$ implies $S$ is compact and $i=2$ implies $X-S$ is either open or compact.
    Let $(i, S) \leq (j,T)$ if $i\leq j$ and $S\subseteq T$ or if $i=0$ and $j=2$.
    For $0 \leq i \leq 2$, let $M_i'$ denote the subset $\{(j,S) \in M' | j=i\} \subseteq M'$.
\end{definition}
\begin{remark}
    Observe that we have identifications $M_0'=M_0\cong \K(X)$, $M_2' \cong \Open(X)^{op} \cup \K(X)^{op}$ and $M_1'=M_1$ is the powerset poset of $X$.
\end{remark}
We will prove Theorem \ref{KVerdierDuality} as a simple corollary of the following proposition.
\begin{proposition}[{\cite[Proposition 5.5.5.7]{HA}}]\label{5.5.5.7}
    Let $X$ be a locally compact Hausdorff space, $\C$ be a stable $\infty$-category with small limits and colimits and $M$ be as in \ref{5.5.5.5}.
    Let $F \colon M \to \C$ be a functor.
    Then the following conditions are equivalent:
    \begin{enumerate}
        \item The restriction $F|_{M_0}^{op}$ determines a $\K$-sheaf $\K(X)^{op} \to \C^{op}$, the restriction $F|_{M_1}$ is zero, and $F$ is left Kan extended from $M_0 \cup M_1$.
        \item The restriction $F|_{M_2}$ determines a $\K$-sheaf $\K(X)^{op} \to \C$, the restriction $F|_{M_1}$ is zero, and $F$ is right Kan extended from $M_1 \cup M_2$.
    \end{enumerate}
\end{proposition}
The proof of this is a bit long, so we will split the theorem into a few lemmas, keeping the notation of the theorem statement.
\begin{lemma}
    It is enough to show that condition $2.$ implies condition $1.$.
\end{lemma}
\begin{proof}
    Observe that the map $(i,S) \mapsto (2-i, X-S)$ is an order-reversing bijection $M\to M$ which is moreover self-inverse.
    Now, it is enough to observe that when $\C$ is stable with small limits and colimits, so is $\C^{op}$ and we can safely swap $\C$ and $\C^{op}$.
\end{proof}
\begin{definition}[{\cite[Definition 5.5.5.9]{HA}}]
    Let $X$ be a locally compact Hausdorff space and $\C$ a pointed $\infty$-category with small limits and colimits.
    For a sheaf $\F \in \Shv(X;\C)$ and $K$ compact we denote by $\Gamma_K(X;\F)$ the fiber product $\F(X)\times_{\F(X-K)}0$.
    For $U$ open, we denote by $\Gamma_c(U;\F)$ the filtered colimit $\colim_{K \in\K(X)_{/U}}\Gamma_K(X;\F)$ where $K$ ranges over all compact subsets of $U$.
    The construction $U \mapsto \Gamma_c(U;\F)$ determines a functor
    \[
        \Gamma_c(\blank ;\F) \colon \Nerve(\Open(X)) \to \C .
    \]
\end{definition}
\begin{lemma}
    Let $\D$ denote the full subcategory of $\Fun(M', \C)$ spanned by those functors $F$ that satisfy the following conditions:
    \begin{enumerate}
        \item $F|_{M_2}$ is a $\K$-sheaf on $X$.
        \item $F|_{M'_2}$ is a right Kan extension of $F|_{M_2}$.
        \item $F|_{M'_1}$ is zero.
        \item $F|_{M'}$ is a right Kan extension of $F|_{M'_1 \cup M'_2}$.
    \end{enumerate}
    Then any $F\in \D$ can be restricted to a sheaf $\F \in \Shv(X;\C)$ and is given by the fiber $\Gamma_K(X;\F)$ when restricted to $K\in M_0$.
\end{lemma}
\begin{proof}
    Observe that we have a bijection between $\Open(X)^{op}$ and the partially ordered set of closed subsets of $X$ by sending $\Open(X) \ni U \mapsto (X - U)$, and we have a natural inclusion $\Open(X)^{op} \hookrightarrow M_2'$.
    By Theorem \ref{7.3.4.9} we can restrict functors in $\Fun(\Open(X)^{op}, \C)$ to $\Shv(X;\C)$, so we can also restrict functors $M_2'\to \C$ and even functors in $\D$ to sheaves on $\Open(X)^{op}$.
    Let $\F$ be the sheaf obtained by restricting $F$.
    Define $\varphi\colon\Nerve(M_0) \to \Fun(\Delta^1\times \Delta^1, \Nerve(M'))$ by sending an object $(0,K) \in M_0$ to the diagram
    \[\begin{tikzcd}
            {(0,K)} && {(1,K)} \\
            \\
            {(2,\emptyset)} && {(2,K)}
            \arrow[from=1-1, to=1-3]
            \arrow[from=1-3, to=3-3]
            \arrow[from=3-1, to=3-3]
            \arrow[from=1-1, to=3-1].
        \end{tikzcd}\]
    We can regard $\varphi(0,K)$ as a map $i: \Lambda_2^2 \to (M_1'\cup M_2')_{(0,K)/}$:
    \[\begin{tikzcd}
            && a &&&&&& {(0,K)\to(1,K)} \\
            &&& {} &&& {} \\
            b && c &&&& {(0,K)\to(2,\emptyset)} && {(0,K)\to(2,K)}
            \arrow[from=1-9, to=3-9]
            \arrow[from=3-7, to=3-9]
            \arrow[from=3-1, to=3-3]
            \arrow[from=1-3, to=3-3]
            \arrow[from=2-4, to=2-7]
        \end{tikzcd}\]
    Here we have abused notation to write the fiber product $M'_{(0, K) /} \times_{M'}M_1^{\prime} \cup M_2^{\prime}$ as $ (M_1'\cup M_2')_{(0,K)/}$. By \ref{superlemma} $i$ is limit-cofinal if and only if for every $(m,A) \in (M_1'\cup M_2')_{(0,K)/}$ the fiber product
    \[\begin{tikzcd}
            {\mathrm{PB}} && {\left((M_1'\cup M_2')_{(0,K)/}\right)_{/(m,A)}} \\
            \\
            {\Lambda_2^2} && {(M_1'\cup M_2')_{(0,K)/}}
            \arrow[from=1-1, to=1-3]
            \arrow["j", hook',from=1-3, to=3-3]
            \arrow["i", hook,from=3-1, to=3-3]
            \arrow[from=1-1, to=3-1]
            \arrow["\lrcorner"{anchor=center, pos=0.125}, draw=none, from=1-1, to=3-3]
        \end{tikzcd}\]
    is weakly contractible.
    As we have partially ordered sets,
    %\prepp{Partially ordered sets only have arrows at the ``bottom'' level or you could say that since partially ordered sets have no isofibrations there's none to lift so any homotopy pullbacks are trivially also strict pullbacks? Also something about pullback of fully faithful being fully faithful is probably needed.\url{https://ncatlab.org/nlab/show/homotopy+pullback\#HomotopyFiberCharacterization}
    %   Corollary 3.4 here might be needed.}
    %\TODO{Make above remark precise.} the pullback is exactly the subset of $\Lambda_2^2 \times \left(\Nerve(M_1' \cup M_2')_{(0,K)/}\right)_{/(m,A)}$ containing those $(t, (r, B))$ such that $i(t)=j(r,B)$.
    %Objects in $\left(\Nerve(M_1' \cup M_2')_{(0,K)/}\right)_{A/}$ are of the form $(n, B) \leq (m, A)$.
    $j$ is just the inclusion given by sending objects $((0,K) \leq (r, B) \leq (m,A))$ in $\left((M_1' \cup M_2')_{(0,K)/}\right)_{/(m,A)}$ to objects $((0,K) \leq (r,B))$ in ${(M_1'\cup M_2')_{(0,K)/}}$.
    Since $i(a)=(2,\emptyset), i(b)=(1,K)$ and $i(c)=(2,K)$ and the pullback of a mono is mono, $\mathrm{PB}$ has to be a subcategory of
    \[\begin{tikzcd}
            && {(b,(1,K))} \\
            \\
            {(a,(2,\emptyset))} && {(c,(2,K))}
            \arrow[from=3-1, to=3-3]
            \arrow[from=1-3, to=3-3].
        \end{tikzcd}\]
    Observe that such a subcategory fails to be contractible only if $(m,A)$ is chosen such that the pullback is either empty or consists of two disjoint objects.
    If $r=1$ we know $(1,K) \leq (1,B)$ and have no arrows from $(2,\emptyset)$ or $(2,K)$ to $(1,B)$.
    If $r=2$ we must have $(2,\emptyset) \leq (2,K) \leq (2,B)$, so the pullback is always weakly contractible.
    %Observe that by condition $4.$ we can write $F'(t, T)$ as $\lim_{(s, S) \in (t, Y)}F'|_{\Nerve(M_1'\cup M_2')}(s,S)$.
    By condition $4.$, $F|_{M'}$ is right Kan extended from $F|_{M_1' \cup M_2'}$, which by definition \ref{KanDef} means
    \[
        (M_1' \cup M_2')^{\triangleleft}_{(0,K)/} \hookrightarrow M'^{\triangleleft}_{(0,K)/}\rightarrow M' \xrightarrow{F} \C
    \]
    is a limit diagram.
    In other words, we have
    \[
        \lim_{(M_1' \cup M_2')_{(0,K)/}}F = F(0,K)
    \]
    and by the limit-cofinality of $i$ we get
    \[
        F(0,K) = \lim_{(M_1' \cup M_2')_{(0,K)/}}F = \lim_{\Lambda_2^2}(F\circ i) = \lim F((2,\emptyset) \rightarrow (2,K) \leftarrow (1,K))
    \]

    which means condition $4.$ is equivalent to requiring that $F$ composed with $\varphi(0,K)$ yields another pullback diagram
    \[\begin{tikzcd}
            {F(0,K)} && {F(1,K)} \\
            \\
            {F(2,\emptyset)} && {F(2,K)}
            \arrow[from=1-1, to=1-3]
            \arrow[from=1-3, to=3-3]
            \arrow[from=3-1, to=3-3]
            \arrow[from=1-1, to=3-1]
            \arrow["\lrcorner"{anchor=center, pos=0.125}, draw=none, from=1-1, to=3-3]
        \end{tikzcd}\]
    Observe now that by condition $3.$ $F(1,K)=0$ and hence
    \[
        F(0,K)\simeq \fib(F(2,\emptyset) \to F(2,K)).
    \]
    Recall that we defined $\F$ as the restriction of $F$ to $\Shv(X;\C)$ by identifying open sets $U$ with their complements.
    This means that $F(2,\emptyset) = \F(X)$ and $F(2,K) = \F(X-K)$ which in turn means that
    \[
        F(0,K)\simeq \fib(F(2,\emptyset) \to F(2,K)) \simeq \fib(\F(X) \to \F(X-K)) = \Gamma_K(X;\F)
    \]
    which completes the proof that $F|_{M_0}(K)$ is given by $\Gamma_K(X;\F)$.
\end{proof}
\begin{corollary}
    In particular this means that any $F\colon M \to \C$ satisfying condition $2.$ in \ref{5.5.5.7} can be restricted to a sheaf $\F\in\Shv(X;\C)$ which is given by $\Gamma_K(X;\F)$ when restricted to $K \in M_0$.
\end{corollary}
\begin{proof}
    Observe first that $M$ is a full subcategory of $M'$ which means we can extend $F$ to a functor $F'\in\MAP(M',\C)$ by fully faithful Kan extension\ref{KanFullyFaithInfty}.
    To show this extension lies in $\D$, we must show that it satisfies the four conditions of $\D$:
    \begin{enumerate}
        \item We have $F'|_{M_2} \in \Shv_{\K}(X;\C)$ by assumption.
        \item We must show $F'|_{M_2}=F|_{M_2}$ is a right Kan extension of $F'|_{M_2'}$.
              As we defined $F'$ to be a right Kan extension of $F$ to $M'$ and $M$ is equal to $M'$ everywhere but on $M_2'$, it is enough to show that the Kan extension only depends on $M_2$.
              In other words, we must show that for any $x \in M_2'$ the map $(M_2)_{x/} \to M_{x/}$ is limit-cofinal, but observing that we can have no maps from $x$ to something in $M - M_2$ we see that this is actually an isomorphism.
        \item We have $F'|_{M_1'}=0$ by assumption.
        \item We must show $F'|_{M'}$ is a right Kan extension of $F'|_{M'_1 \cup M'_2}$.
              Observing that the second condition is equivalent to $F'|_{M'_1 \cup M'_2}$ being a right Kan extension of $F'|_{M_1 \cup M_2}$, we can use transitivity of Kan extensions (\ref{4.3.2.8}) to see that $F'|_{M'}$ is a right Kan extension of $F'|_{M'_1 \cup M'_2}$ if and only if $F'|_{M}$ is a right Kan extension of $F'|_{M_1 \cup M_2}$ which is true by assumption.
    \end{enumerate}
    This shows that $F'$ belongs to $\D$ so the Lemma implies $F|_{M_0} = F'|_{M_0} = \Gamma_K(X;\F)$ for $K\in M_0$.
\end{proof}
\begin{lemma}
    Let $\G = F|_{M_0}$.
    Then $\G^{op}\colon M_0^{op} \to \C^{op}$ is a $\K$-sheaf valued in $\C^{op}$.
\end{lemma}
\begin{proof}
    Here $\G^{op}$ does the same on objects as $\G$ but we think of it as a functor $M_0^{op}\to \C^{op}$.
    We must show that it satisfies the following three properties:
    \begin{enumerate}
        \item $\G(\emptyset)$ is a zero object, because $\G(\emptyset)=\fib(\F(X)\to \F(X-\emptyset))=0$.
        \item For any compact subsets $K$ and $K'$ of $X$, we must show the following diagram is a pullback:
              \[\begin{tikzcd}
                      {\G(K\cap K')} && {\G(K')} \\
                      \\
                      {\G(K)} && {\G(K\cup K')}
                      \arrow[from=1-1, to=3-1]
                      \arrow[from=3-1, to=3-3]
                      \arrow[from=1-1, to=1-3]
                      \arrow[from=1-3, to=3-3]
                      \arrow["\lrcorner"{anchor=center, pos=0.125}, draw=none, from=1-1, to=3-3]
                  \end{tikzcd}\]
              Observe that the diagram can be identified with the fiber of the map
              \[\begin{tikzcd}
                      {\F(X)} && {\F(X)} && {\F(X-(K\cap K'))} && {\F(X-K')} \\
                      && {} && {} \\
                      {\F(X)} && {\F(X)} && {\F(X-K)} && {\F(X-(K\cup K'))}
                      \arrow[from=1-5, to=3-5]
                      \arrow[from=3-5, to=3-7]
                      \arrow[from=1-5, to=1-7]
                      \arrow[from=1-7, to=3-7]
                      \arrow["\lrcorner"{anchor=center, pos=0.125}, draw=none, from=1-5, to=3-7]
                      \arrow[from=2-3, to=2-5]
                      \arrow[from=1-3, to=3-3]
                      \arrow[from=1-1, to=1-3]
                      \arrow[from=1-1, to=3-1]
                      \arrow[from=3-1, to=3-3]
                      \arrow["\lrcorner"{anchor=center, pos=0.125}, draw=none, from=1-1, to=3-3]
                  \end{tikzcd}\]
              As $\F$ is a sheaf, this is a map between pullbacks, so our diagram is also a pullback.
        \item For any compact subset $K$ of $X$ we must show that the map $\theta\colon \G(K)\to \lim_{K\Subset K'}\G(K')$ is an equivalence in $\C$.
              Observe now that $\theta$ gives us a map between two fiber sequences
              \[\begin{tikzcd}
                      {\G(K)} && {\lim_{K\Subset K'}\G(K')} \\
                      \\
                      {\F(X)} && {\lim_{K\Subset K'}\F(X)} \\
                      \\
                      {\F(X-K)} && {\lim_{K\Subset K'}\F(X-K')}
                      \arrow["\theta", from=1-1, to=1-3]
                      \arrow["{\theta'}", from=3-1, to=3-3]
                      \arrow["{\theta''}", from=5-1, to=5-3]
                      \arrow[from=1-1, to=3-1]
                      \arrow[from=3-1, to=5-1]
                      \arrow[from=1-3, to=3-3]
                      \arrow[from=3-3, to=5-3]
                  \end{tikzcd}\]
              Since the partially ordered set $\{K'\in \K(X) | K\Subset K'\}$ is filtered, it is weakly contractible and hence $\theta'$ is an equivalence.
              Since $\F$ is a sheaf and the set $\{X-K' | K \Subset K'\}$ is a covering sieve on $X-K'$, $\theta''$ is also an equivalence.
              As we have shown that $\theta'$ and $\theta''$ are equivalences, $\theta$ must also be an equivalence, and we have shown that $\G^{op}$ determines a $\K$-sheaf $(\K(X)^{op} \to \C^{op})$.
    \end{enumerate}
\end{proof}
To complete the proof of Proposition \ref{5.5.5.7}, we must show the following lemma:
\begin{lemma}
    $F$ is left Kan extended from $F|_{M_0\cup M_1}$.
\end{lemma}
\begin{proof}
    Let $M''=\{(i,S) \in M_0 \cup M_1 \mid (i,S)\in \K(X)\}$.
    We can observe that $F|_{M_0 \cup M_1}$ is left Kan extended from $F|_{M''}$ ($F$ is zero on $M_1$).
    By Proposition \ref{4.3.2.8} it is enough to show that $F$ is a left Kan extension of $F|_{M''}$, and for this it is enough to check at every $(2,S) \in M_2$.
    We will instead show that $F|_{M'' \cup M_2'}$ is a left Kan extension of $F|_{M''}$, and for this we define
    \[B:=\{(2, X- U) \subseteq M_2' \mid U \in \Open(X) | \overline{U} \in \K(X)\} .\]
    By transitivity of Kan extensions (Proposition \ref{4.3.2.8}) it is enough to show that
    \begin{enumerate}[label=(\alph*)]
        \item $F'|_{M'' \cup M_2'}$ is a left Kan extension of $F|_{M'' \cup B}$ and
        \item $F|_{M'' \cup B}$ is a left Kan extension of $F|_{M''}$.
    \end{enumerate}
    We first show (a):
    First observe that $M''$ and $M_2'$ are disjoint so it is enough to check that for every

    $(2,X-K)\in M_2' - B$, the composite
    \[
        (M'' \cup B)^{\triangleright}_{/(2,X-K)} \hookrightarrow  (M'' \cup M_2')^{\triangleright}_{/(2,X-K)} \to (M''\cup M_2') \to \C
    \]
    is a colimit diagram.
    According to Theorem \ref{superlemma} the inclusion $B_{/(2,X-K)}\subseteq (M''\cup B)_{/(2,X-K)}$ is colimit-cofinal if we can show that the pullback
    \[\begin{tikzcd}
            PB && {\left((M''\cup B)_{/(2,X-K)}\right)_{/(2,X-U)}} \\
            \\
            {\left(B_{/(2,X-K)}\right)_{/(2,X-U)}} && {(M''\cup B)_{/(2,X-K)}}
            \arrow[from=1-1, to=1-3]
            \arrow[from=1-3, to=3-3]
            \arrow[from=3-1, to=3-3]
            \arrow[from=1-1, to=3-1]
        \end{tikzcd}\]
    is weakly contractible for all $(2,X-U) \in (M''\cup B)$.
    As this is just the partially ordered set $\{(i,S)\in B \mid (i,S) \leq (2,X-U) \leq (2,X-K)\}$ it is weakly contractible by the usual argument (it is nonempty and stable under finite unions, hence filtered).
    This means that it is enough to show that $F'|_{M_2'}$ is left Kan extended from $B$.
    Assumption $2.$ says that $F|_{M_2}$ determines a $\K$-sheaf, $F|_{M_1}=0$ and that $F$ is a right Kan extension from $M_1 \cup M_2$.
    Identifying $M_2 = \{(2, S) | (X-S) \in \K(X)\}$ with $\K(X)^{op}$ we see that we are in the situation of Theorem \ref{7.3.4.9}.
    As $M_2' = \{(2,S) | (X-S) | \in \Open(X) \cup \K(X)\}$ we can identify it with $(\Open(X) \cup \K(X))^{op}$ and by Theorem \ref{7.3.4.9} we get that $F|_{M_2'}$ is a left Kan extension of $F|_{\Open(X)^{op}}$.
    By observing that for a $K \in \K(X)$ the collection of open neighborhoods of $K$ with compact closure is colimit-cofinal in the collection of all open neighborhoods of $K$ in $X$ we get that $F|_{M_2'}$ is furthermore left Kan extended from $B$, which was what we wanted to show.
    \[\begin{tikzcd}
            B && {\Open(X)^{op}} && \C \\
            \\
            &&& {M_2'}
            \arrow[hook, from=1-3, to=3-4]
            \arrow[from=1-3, to=1-5]
            \arrow[from=3-4, to=1-5]
            \arrow["i", hook, from=1-1, to=1-3]
        \end{tikzcd}\]
    Here we have used that we calculate Kan extensions as colimits, so $i$ being colimit-cofinal over some fixed $K$ means restricting the colimit from $\Open(X)^{op}$ back to $B$ is an equivalence.


    We now show (b):
    % When using 4.2.3 we are probably using 4.2.3.9 and 4.2.3.10.
    Fix $U \in \Open(X)$ such that $\overline{U} \in \K(X)$. By \ref{KanDef} we want to show that $F(2, X-U)$ is a colimit of the diagram $F|_{M''_{/(2,X-U)}}$.
    For $K \in \K(X)$ denote by $M_K''$ the subset of $M''$ consisting of pairs $(i,S)$ such that $(0,K)\leq (i,S) \leq (2,X-U)$.
    Now observe that $M''_{/(2,X-U)}$ is a filtered colimit of the simplicial sets $M''_K$ over ${K\in \K(X)_{U/}}$.
    By \cite[Remark 4.2.3.9.]{HTT} and \cite[Corollary 4.2.3.10.]{HTT} we can identify $\colim(F|_{M''})_{/(2,X-U)}$ with the filtered colimit of the diagram $\{\colim(F|_{M_K''})\}_K$.
    This means that we are reduced to showing that for every $K \in \K(X)_{U/}$, $F$ exhibits $F(2,X-U)$ as a colimit of $F|_{M_K''}$.
    By Theorem \ref{superlemma} the diagram
    \[\begin{tikzcd}
            {(0,K-U)} && {(1,K-U)} \\
            \\
            {(0,K)}
            \arrow[from=1-1, to=1-3]
            \arrow[from=1-1, to=3-1]
        \end{tikzcd}\]
    is limit-cofinal in $\Nerve(M_K'')$ and hence it is enough to show that
    \[\begin{tikzcd}
            {F(0,K-U)} && {F(1,K-U)} \\
            \\
            {F(0,K)} && {F(2,X-U)}
            \arrow[from=1-1, to=1-3]
            \arrow[from=1-1, to=3-1]
            \arrow[from=1-3, to=3-3]
            \arrow[from=3-1, to=3-3]
        \end{tikzcd}\]
    is a pushout in $\C$.
    We will show this by considering the larger diagram
    \[\begin{tikzcd}
            {F(0,K-U)} && {F(1,K-U) = 0} \\
            \\
            {F(0,K)} && Z && {F(1,K) = 0} \\
            \\
            {F(2,\emptyset)} && {F(2,K-U)} && {F(2,K)} \\
            \\
            && {F(2,X-U)} && {F(2,X)}
            \arrow[from=1-1, to=1-3]
            \arrow[from=1-1, to=3-1]
            \arrow[from=1-3, to=3-3]
            \arrow[from=3-1, to=3-3]
            \arrow[from=3-1, to=5-1]
            \arrow[from=5-1, to=5-3]
            \arrow[from=3-3, to=5-3]
            \arrow[from=3-3, to=3-5]
            \arrow[from=5-3, to=5-5]
            \arrow[from=3-5, to=5-5]
            \arrow[from=5-3, to=7-3]
            \arrow[from=7-3, to=7-5]
            \arrow[from=5-5, to=7-5]
            \arrow["\lrcorner"{anchor=center, pos=0.125}, draw=none, from=3-3, to=5-5]
        \end{tikzcd}\]
    where we already know that the middle composite rectangle is a pullback (we have shown $F(0,K)$ to be the fiber of the map $F(2,\emptyset) \to F(2,K)$), so the middle left square is also a pullback.
    As we have shown $F(0,K-U) = \fib(F(2,\emptyset) \to F(2,K-U))$ the left vertical composite rectangle is also a pullback, so the upper left must be as well.


    Finally, we use that $\C$ is stable by noting that the upper left square is also a pushout which means that we are done if we can show that $Z$ is equivalent to $F(2,X-U)$.
    By the corollary, the bijection between $\Open(X)^{op}$ and the poset of closed subsets of $X$ induces a restriction of $F$ to a sheaf $\F$.
    As $F(1,K)=0$ and $F(2,X)=\F(\emptyset)=0$ we have an equivalence $F(1,K)\to F(2,K) \to F(2,X)$ which means that if we can show the composite square
    \[\begin{tikzcd}
            Z && {F(1,K)=0} \\
            \\
            {F(2,X-U)} && {F(2,X)=\F(\emptyset)=0}
            \arrow[from=1-1, to=1-3]
            \arrow[from=1-3, to=3-3]
            \arrow[from=3-1, to=3-3]
            \arrow[from=1-1, to=3-1]
        \end{tikzcd}\]
    is a pullback, we have shown the desired equivalence $Z \to F(2,X-U)$ using the fact that pullback along an equivalence is again an equivalence.

    To complete the proof it is therefore enough to show that the lower right square is a pullback.
    Replacing $F$ by $\F$ we get
    \[\begin{tikzcd}
            {\F((X-K) \cup U)} && {\F(X-K)} \\
            \\
            {\F(U)} && {\F(\emptyset)}
            \arrow[from=1-1, to=1-3]
            \arrow[from=1-3, to=3-3]
            \arrow[from=3-1, to=3-3]
            \arrow[from=1-1, to=3-1]
        \end{tikzcd}\]
    which is a pullback because $\F$ is a sheaf ($U$ and $X-K$ are disjoint).\qedhere
\end{proof}
We can now prove Verdier Duality (Theorem \ref{VerdierDuality}):
\begin{proof}
    Let $\mathcal{E}(\C) \subseteq \Fun(M)$ be the full subcategory spanned by those functors satisfying the conditions of Proposition \ref{5.5.5.7} and observe that the inclusions $M_0 \hookrightarrow M \hookleftarrow M_2$ give restrictions
    \[
        \Shv_{\K}(X;\C^{op}) \xleftarrow{\theta} \mathcal{E}(\C)^{op} \xrightarrow{\theta'} \Shv_{\K}(X;\C)^{op}.
    \]
    %\prepp{Now Lurie cites \cite{HTT}[4.3.2.15]to say the restrictions are trivial Kan fibrations, but I want to say the following instead (I think it is almost the same thing):}
    Because we Kan extend along inclusions of full subcategories these are equivalences of $\infty$-categories by Proposition \ref{KanFullyFaithInfty}.
    %\prepp{Note that $F\in \mathcal{E}(\C)$ means $F|_{M_0}\in \Shv_{\K}(X;\C^{op})$, $F|_{M_0 \cup M_1} = RKAN(F|_{M_0})$ and $F=LKAN(F|_{M_0 \cup M_1})$ .}
    This proves Theorem \ref{KVerdierDuality} and by Corollary \ref{7.3.4.10} we have shown Theorem \ref{VerdierDuality}.
\end{proof}
\begin{remark}\label{GammaKXisGammaKU}
    Observe that for $K$ a compact subset of an open subset $U$, $\F$ is a sheaf means that we have pullbacks:
    \[\begin{tikzcd}
            {\fib(f)} && {\Gamma(X;\F)} && {\Gamma(U;\F)} \\
            \\
            0 && {\Gamma(X-K;\F)} && {\Gamma(U-K;\F)}
            \arrow[from=1-3, to=1-5]
            \arrow["f", from=1-3, to=3-3]
            \arrow[from=3-3, to=3-5]
            \arrow["g", from=1-5, to=3-5]
            \arrow["\lrcorner"{anchor=center, pos=0.125}, draw=none, from=1-3, to=3-5]
            \arrow[from=1-1, to=3-1]
            \arrow[from=3-1, to=3-3]
            \arrow[from=1-1, to=1-3]
            \arrow["\lrcorner"{anchor=center, pos=0.125}, draw=none, from=1-1, to=3-3]
        \end{tikzcd}\]
    As the composition of pullbacks is again a pullback we get $\Gamma_K(X;\F)=\fib(f)=\fib(g)=\Gamma_K(U;\F)$.
\end{remark}
\begin{proposition}[{\cite[Proposition 5.5.5.10]{HA}}]
    Let $X$ be a locally compact Hausdorff space and $\C$ a stable $\infty$-category with small limits and colimits.
    Then the equivalence of $\infty$-categories
    \[
        \mathbb{D} \colon \Shv(X;\C)^{op} \simeq \Shv(X;\C^{op}).
    \]
    given in Theorem \ref{VerdierDuality} is given by $\mathbb{D}(\F)(U) = \Gamma_c(U;\F)$.
\end{proposition}
\begin{proof}
    It follows from the proof of Theorem \ref{7.3.4.9} that the equivalence
    \[
        \theta\colon \Shv_{\K}(X;\C^{op})^{op} \simeq \Shv(X;\C^{op})^{op}
    \]
    is given by the formula $\theta(\F)(U) = \colim_{K\subseteq U}\F(K)$, where the colimit is taken in $\C$.
    Let $\psi\colon \Shv(X;\C) \to \Shv_{\K}(X;\C)$ be the equivalence of Corollary \ref{7.3.4.10} and let $\psi'$ be the equivalence $\Shv_{\K}(X;\C) \to \Shv_{\K}(X;\C^{op})^{op}$ of Theorem \ref{KVerdierDuality}.
    Composing, we get a string of equivalences
    \[
        \mathbb{D}^{op} \colon \Shv(X;\C) \xrightarrow{\psi} \Shv_{\K}(X;\C) \xrightarrow{\psi'} \Shv_{\K}(X;\C^{op})^{op} \xrightarrow{\theta} \Shv(X;\C^{op})^{op}.
    \]
    % \[\begin{tikzcd}
    % 	{\Shv(X;\C)} && {\Shv(X;\C^{op})^{op}} \\
    % 	\\
    % 	{\Shv_{\K}(X;\C)} && {\Shv_{\K}(X;\C^{op})^{op}}
    % 	\arrow["{\D^{op}}", from=1-1, to=1-3]
    % 	\arrow["\psi"', from=1-1, to=3-1]
    % 	\arrow["{\psi'}"', from=3-1, to=3-3]
    % 	\arrow["\theta"', from=3-3, to=1-3]
    % \end{tikzcd}\]
    Let $\D$ be as in the proof of Lemma \ref{5.5.5.7}.
    By Theorem \ref{7.3.4.9} the restriction $\D\to \Fun(\Open(X)^{op}, \C)$ is a categorical equivalence onto $\Shv(X;\C)$, since we Kan extend from the full subcategory $\Shv(X;\C)$.
    \newline
    In the other direction we restrict $\D \to \Fun(M_0, \C) \simeq \Fun(\K(X), \C) \simeq \Fun(\K(X)^{op}, \C^{op})^{op}$ and $\psi' \circ \psi$ is given by the composition $\Shv(X;\C) \to \D \to \Fun(\K(X)^{op}, \C^{op})^{op}$, and as we saw in the proof of Lemma \ref{5.5.5.7}, restriction from $\D$ to functors from $M_0$ is given by $\Gamma_K(X;\F)$.
    This means that $\psi' \circ \psi \colon \F \mapsto \left( K \mapsto \Gamma_K(X;\F) \right)$ so by Remark \ref{GammaKXisGammaKU} we have
    \[
        (\theta \circ \psi' \circ \psi)(\F)(U) = \colim_{K\subseteq U}(\Gamma_K(X;F)) = \colim_{K\subseteq U}(\Gamma_K(U;F)) = \Gamma_c(U;\F).\qedhere
    \]
\end{proof}
\begin{remark}
    This is the infinity-categorical generalization of the classical fact that conjugation by Verdier duality exchanges cohomology and cohomology with compact support.
    The construction of $\Gamma_c(U;\F)$ above really is analogous to the construction
    \[
        \Gamma_c(X ; \F) := \{ s \in \Gamma(X; \F) \mid s\text{ has compact support}\}
    \]
    from classical sheaf theory in $1$-categories.
    Also, recall that in the introduction of this chapter we defined direct image with proper support as the functor
    \[
        \Gamma(U; f_!\F) : = \{ s \in \Gamma(U; f_*\F) \mid s \text{ has compact support}\}.
    \]
    and said Verdier duality is the existence of a right adjoint $f^!$.


    In \cite{Volpe}, Marco Volpe shows that one can extend the classical six functor formalism for sheaves on locally compact Hausdorff spaces to sheaves with values in any closed symmetric monoidal $\infty$-category which is stable and bicomplete, and Lurie's Verdier duality (Theorem \ref{VerdierDuality}) is central in proving the $\infty$-categorical adjunction.
    Volpe uses the functors $f_*$ and $f^*$ on $\C^{op}$, which takes cosheaves to cosheaves, to obtain the functors $f_!$ and $f^!$ and uses Theorem \ref{VerdierDuality} to transfer the adjunction to the level of sheaves.
    More precisely, he shows an equivalence $\operatorname{CoShv}(X ; \mathrm{Sp}) \otimes \C \simeq \operatorname{CoShv}(X ; \C)$ and uses Verdier duality to obtain an equivalence $\Shv(X;\mathrm{Sp})\otimes \C \simeq \Shv(X;\C)$.
    With this equivalence, finding $f^*$ is rephrased in terms of sheaves with presentable coefficients, and these are easier to work with due to the existence of sheafification.
    In particular, it is known that for a proper map $f\colon X\to Y$ of locally compact Hausdorff spaces, the functor $f_*^{\mathrm{Sp}}:\Shv(X;\mathrm{Sp})\to \Shv(Y;\mathrm{Sp})$ admits a left adjoint $f_{\mathrm{Sp}}^*$ and one then obtains the desired adjunction by observing that the equivalence $\Shv(X;\mathrm{Sp})\otimes \C \simeq \Shv(X;\C)$ allows us to identity $f_*^{\mathrm{Sp}}\otimes \C$ with $f_*\colon \Shv(X;\C) \to \Shv(Y;\C)$ and $f_{\mathrm{Sp}}^* \otimes \C$ with $f^*$.
\end{remark}
\begin{remark}
    As we saw in the introduction to this chapter, classical Verdier duality can be used to give a proof of Poincaré duality, and in \cite[Section 5.5.6]{HA} Jacob Lurie uses his version of Verdier duality to prove a version of what he calls non-abelian Poincaré duality.
\end{remark}
\end{document}