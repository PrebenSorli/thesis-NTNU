\documentclass[../../thesis.tex]{subfiles}

\begin{document}
What Lurie \cite{HTT} calls $\infty$-categories were originally called restricted Kan complexes by Boardman and Vogt \cite{BoardmanVogt}, but without the intent of using them for $\infty$-categories.
The first development of such a theory was done by Joyal in \cite{Joyal}, who called them quasicategories.
As most of this thesis follows Lurie's works very closely, we will follow his convention and use the name $\infty$-categories.\footnote{It should be noted that in other sources ``$\infty$-categories'' might refer to other models than the one we use.}
While \cite{HTT} gives a good introduction to $\infty$-categories extending on the work of Joyal, his web-project \cite{kerodon} reworks a lot of the foundations, and we take a lot of inspiration from this presentation.
We have also used Charles Rezk's great notes \cite{Rezk} on $\infty$-categories.
\section{Simplicial sets}
Originally, simplicial sets were used to rephrase the homotopy theory of spaces in combinatorial terms.
There are many good introductions to simplicial sets, depending on what you want to use them for, but Friedman's \cite{friedman2021elementary} was enlightening for the author of this thesis.
For algebraic topologists, Peter May's \cite{MAY} is a good introduction to semi-simplicial topology.
\begin{definition}
    Usually denoted by $\Delta$, the simplex category or the simplicial category is the category with linearly ordered sets $[n]=\{0,1,2,\cdots ,n\}$ as its objects and order-preserving maps between them as its morphisms.
    That is, for a map $\varphi\colon[m] \to [n]$ we have that $0 \leq \varphi(i)\leq \varphi(j)\leq n$ for each $0 \leq i \leq j \leq m$.
\end{definition}
We denote by $\delta^i$ the elementary face operator $[n-1] \to [n]$ and by $\sigma^i$ the elementary degeneracy operator $[n+1] \to [n]$ given by
\[
    \begin{aligned}
        \delta^i(j) =
        \begin{cases}
            j   & \text{ if } j < i    \\
            j+1 & \text{ if } j \geq i
        \end{cases}, \quad
        \sigma^i(j) =
        \begin{cases}
            j   & \text{ if } j \leq i \\
            j-1 & \text{ if } j > i
        \end{cases}
    \end{aligned}
\]
\begin{remark}
    All morphisms in $\Delta$ are finite compositions of such morphisms.
\end{remark}
\begin{definition}
    We define the category $\sset \colon = \Fun(\Delta^{op}, \SET)$ (also denoted $\SET_{\Delta}$ by Lurie) of simplicial sets as $\SET$-valued presheaves on $\Delta$, i.e. functors $\Delta^{op}\to\SET$.
\end{definition}
Let $X \in \sset$. We will denote by $X_n$ the set $X([n])$ of $n$-simplices (also called $n$-cells) of $X$.
We define the standard $n$-simplex as $\Delta^n:=y([n])$ where $y$ is the Yoneda embedding, meaning $\Delta^n$ is the presheaf $\Hom_{\Delta}(\blank,[n])$.
By the Yoneda lemma $\Hom_{\Fun(\Delta^{op}, \SET)}(\Delta^n, X) \simeq X_n$, so we can identify each simplex $x\in X_n$ with a map $x\colon\Delta^n\to X$.
This application of the Yoneda lemma is a crucial part of the theory of simplicial sets and we will more often than not consider $n$-simplices of a simplicial set $X$ as maps of simplicial sets instead.
Observe, moreover, that composition with the elementary face operator gives us a map $\Delta^{n-1}\to\Delta^n$.
%by $\delta^i \circ\blank$.
\begin{definition}
    For a simplicial set $X$, we define the face and degeneracy maps
    \[
        d_i: = X(\delta^i): X_{n} \to X_{n-1}, \quad s_i: = X(\sigma^i): X_{n} \to X_{n+1}
    \]
    where both maps are given by composition with $\delta^i$ and $\sigma^i$ respectively.
\end{definition}
\begin{example}
    The standard $0$-simplex $\Delta^0:=\Hom(\blank, [0])$ is a terminal object in $\sset$, meaning it maps any $[m]\in \Delta$ to a singleton.
    This is usually just referred to as the point and denoted $*$.
\end{example}
\begin{example}[{\cite[\href{https://kerodon.net/tag/000P}{Remark 000P}]{kerodon}}]
    Let $X \in \sset$ and suppose we have subsets $T_n\subseteq X_n$ for every $n\geq 0$ such that $d_i(T_n)\subseteq T_{n-1}$ and $s_i(T_n) \subseteq T_{n+1}$.
    Then the collection $\{T_n\}_{n\geq 0}$ is a simplicial set we will call a simplicial subset $T\subseteq X$.
\end{example}
\begin{definition}
    We define the boundary $\partial\Delta^n$ of $\Delta^n$ as the simplicial set
    \[
        (\partial\Delta^n)_m = (\partial\Delta^n)([m]) :=\{\alpha \in \Hom_{\Delta}([m], [n]) | [n] \not\subseteq \mathrm{im}(\alpha)\}.
    \]
    Observe that $\partial\Delta^0=\emptyset$ because every map $[m] \to [0]$ is surjective.
\end{definition}
\begin{definition}
    For $0\leq i \leq n$, we define the horn $\Lambda_i^n$ as the simplicial set
    \[
        (\Lambda_i^n)_m =(\Lambda_i^n)([m]) :=\{\alpha \in \Hom_{\Delta}([m], [n]) | \delta^i[n] \not\subseteq \mathrm{im}(\alpha)\}.
    \]
\end{definition}
Observe that the horn is inside the boundary.
We usually refer to $\Lambda_i^n$ as the $i$th horn in $\Delta^n$ and we will call the horns such that $0<i<n$ the inner horns.
\begin{example}\label{SingExample}
    We define the topological $n$-simplex:
    \[
        \Delta^n_{\mathrm{Top}} : = \{(x_0 , \dots , x_n) \in \mathbb{R}^{n+1} | \sum_{i} x_i = 1, x_i \geq 0\},
    \]
    and for a topological space $X$  we define its singular complex $\mathrm{Sing}(X)$ as the simplicial set with cells
    \[
        [n] \mapsto \Hom_{\mathrm{Top}}(\Delta^n_{\mathrm{Top}}, X).
    \]
\end{example}
\begin{example}\label{NerveDef}
    We define the nerve $\Nerve(\C)$ of a $1$-category $\C$ by
    \[
        \Nerve(\C) : = \Hom_{\mathrm{Cat}}([\blank], \C)
    \]
    where we view the sets $[n]$ as categories (posets with a map $i$ to $j$ whenever $i\leq j$).
    Observe that for any order-preserving morphism $\alpha : [m] \to [n]$ we get a map
    \[
        \Hom_{\mathrm{Cat}}([n], \C) \xrightarrow{\blank \circ \alpha} \Hom_{\mathrm{Cat}}([m], \C)
    \]
    and it is clear that the nerve is a simplicial set with $\Nerve(\C)_n = \Hom_{\mathrm{Cat}}([n], \C)$.
    \newline
    Observe furthermore that for a functor $F: \C \to \D$ we get a simplicial map $\Nerve(F):\Nerve(\C) \to \Nerve(\D)$ by sending $n$-cells $\varphi : [n] \to \C$ in $\Nerve(\C)_n$ to $n$-cells $F(\varphi) \colon [n] \to \D$ in $\Nerve(\D)_n$, so the construction is functorial.
    It should also be clear that the set of objects of $\C$ is identified with the $0$-cells $\Nerve(\C)_0$ and the morphisms with the $1$-cells $\Nerve(\C)_1$.
    Additionally, the $2$-cells $\Nerve(\C)_2$ is in bijection with the set of composable pairs of morphisms in $\C$ and likewise the $n$-cells are the strings of $n$ composable morphisms.
    We will talk more about composition of morphisms in the next section.
\end{example}
\begin{definition}
    We define the connected components $\pi_0X$ of a simplicial set $X$ as the colimit of $X$.
\end{definition}
This is equivalently the quotient of $X_0$ by the equivalence relation on $X_0$ generated by the relation $x \sim y$ if and only if there exists an edge $f \in X_1$ such that $f_0=x$ and $f_1=y$.
\section{$\infty$-categories}
Before we give a precise definition, we will take a closer look at the nerve construction.
Clearly, we want the nerve of a $1$-category to give us an $\infty$-category, and most of this thesis will revolve around nerves of certain poset-categories of topological spaces.
Nerves of categories are not just any ordinary simplicial sets but simplicial sets with some more structure inherited from the underlying $1$-category.
For instance, $1$-categories have composition of morphisms.
Take for example
\[\begin{tikzcd}
        X && Y \\
        & Z
        \arrow["f", from=1-1, to=1-3]
        \arrow["g", from=1-3, to=2-2]
        %\arrow["h"', from=1-1, to=2-2]
    \end{tikzcd}\]
in some ordinary $1$-category $\C$.
This diagram gives us a morphism $\Lambda_1^2\to \Nerve(\C)$ of simplicial sets, but in $\C$ the maps $f$ and $g$ can be composed to a morphism $h: X \to Z$ which in turn gives a unique way to extend the simplicial map $\Lambda_1^2\to \Nerve(\C)$ to a map $\Delta^2 \to \Nerve(\C)$.
If we instead look at the outer horns $\Lambda_0^2$ and $\Lambda_2^2$, we will not necessarily have a way to extend morphisms to $\Delta^2$ in general.
For example the diagram
\[\begin{tikzcd}
        X && Y \\
        & X
        \arrow["g", from=1-3, to=2-2]
        \arrow["\id_X"', from=1-1, to=2-2]
    \end{tikzcd}\]
gives a map $\Lambda_2^2\to \Nerve(\C)$, but extending this to a morphism $\Delta^2 \to \Nerve(\C)$ would amount to finding a right inverse to $g$, which of course is not something we can always do in general, unless $\C$ was a groupoid.
This property of extending a morphism from a horn to the standard $n$-simplex is sometimes also called filling the horn, and we will see that it is a defining property for $\infty$-categories.
In fact, the existence of horn fillings completely classifies those simplicial sets which are nerves of categories:
\begin{proposition}[{\cite[Proposition 1.1.2.2]{HTT}}]
    Let $X \in \sset$.
    Then the following conditions are equivalent:
    \begin{enumerate}
        \item There exists a small category $\C$ with an isomorphism $X \simeq \Nerve(\C)$.
        \item Every inner horn $\Lambda_i^n \to X$ of $X$ can be filled in a unique way.
              Or, in other words, for any solid diagram as below, there is a unique dotted arrow making it commute:
              \[\begin{tikzcd}
                      {\Lambda_i^n} && X \\
                      \\
                      {\Delta^n}
                      \arrow[from=1-1, to=1-3]
                      \arrow[from=1-1, to=3-1]
                      \arrow["{\exists!}"', dashed, from=3-1, to=1-3]
                  \end{tikzcd}\equationQED\]
    \end{enumerate}
\end{proposition}
Simplicial sets which admits extensions for all horn inclusions are called Kan complexes:
\begin{definition}
    A simplicial set $X$ is a Kan complex if it satisfies the following condition:
    For $0 \leq i \leq n$, any map $\sigma_0 \colon \Lambda_i^n \to X$ can be extended to a map $\sigma\colon \Delta^n \to X$.
\end{definition}
\begin{example}
    The singular complex $\mathrm{Sing}(X)$ of a topological space is a Kan complex.
\end{example}
\begin{proposition}
    Nerves of groupoids are Kan complexes.
\end{proposition}
\begin{proof}
    The idea of the proof is that all morphisms are invertible, so all horns can be filled.
\end{proof}
As we saw in the example of a map $\Lambda_2^2\to \Nerve(\C)$ above, whenever there's non-invertible morphisms around some outer horns will be impossible to fill.
This motivates the definition of an $\infty$-category.
The following definition is due to Boardman and Vogt \cite{BoardmanVogt} who defined weak Kan complexes as simplicial sets satisfying what they called the restricted Kan condition:
\footnote{Maybe more commonly known as the weak Kan condition.}:
\begin{definition}[Boardman and Vogt \cite{BoardmanVogt}]
    A simplicial set $X$ is an $\infty$-category if it satisfies the following condition:
    For $0 < i < n$, any map $\sigma_0 \colon \Lambda_i^n \to X$ can be extended to a map $\sigma \colon \Delta^n \to X$.
\end{definition}
This means that any Kan complex is an $\infty$-category, and in particular, so is $\mathrm{Sing}(X)$ for a topological space $X$.
Additionally, observe that the nerve $\Nerve(\C)$ of an ordinary category $\C$ is an $\infty$-category.
Because the nerve functor is fully faithful (see example \ref{NerveKan}), many authors choose to omit its notation altogether.
In many ways, $\infty$-categories behave similarly to ordinary categories, and we will often write about them almost as if they were ordinary categories instead.
For example, we will use the terminology of ordinary category theory and refer to the vertices and edges of our simplicial sets as objects and morphisms in our $\infty$-categories.
There are however some obvious differences between ordinary categories and $\infty$-categories which need adressing before adopting complete $1$-categorical language.
For example, and perhaps most crucially, we have higher-level maps given by simplices of dimension $n\geq 2$.
While we have seen that nerves of categories admit unique horn extensions, this condition is dropped for general $\infty$-categories, and hence composition of morphisms in an $\infty$-category is not necessarily unique but rather unique up to homotopy.
Before we can make this precise, we must define what we mean by homotopy.
\subsection{Homotopy}
\begin{definition}[{\cite[\href{https://kerodon.net/tag/003V}{Definition 003V}]{kerodon}}]
    Let $\C$ be an $\infty$-category and $f,g \colon X\to Y$ morphisms in $\C$.
    We define a homotopy between $f$ and $g$ as a $2$-simplex $\sigma \in \C$ with boundary specified by $d_0(\sigma) = \id_Y, d_1(\sigma) = g$ and $d_2(\sigma)=f$ as illustrated in the diagram
    \[\begin{tikzcd}
            X && Y \\
            & Y
            \arrow["f", from=1-1, to=1-3]
            \arrow["g"', from=1-1, to=2-2]
            \arrow["{\id_Y}", from=1-3, to=2-2].
        \end{tikzcd}\]
    We say $f$ and $g$ are homotopic if such a homotopy $\sigma$ exists.
\end{definition}
\begin{example}
    For a $1$-category $\C$ two morphisms $f,g \colon X \to Y$ in $\C$ are homotopic in $\Nerve(\C)$ if and only if $f=g$.
\end{example}
\begin{proposition}[{\cite[\href{https://kerodon.net/tag/003Z}{Proposition 003Z}]{kerodon}}]
    Let $\C$ be an $\infty$-category and $X,Y$ objects of $\C$.
    Then homotopy is an equivalence relation on the collection of morphisms from $X$ to $Y$ in $\C$.
\end{proposition}
\begin{proof}
    Let $f \colon X \to Y$ be a map in $\C$.
    \newline
    First observe that reflexivity follows from considering the degenerate $2$-simplex $s_1(f)$:
    \[\begin{tikzcd}
            & Y \\
            X && Y
            \arrow["f", from=2-1, to=1-2]
            \arrow["f"', from=2-1, to=2-3]
            \arrow[shift right=1, no head, from=1-2, to=2-3]
            \arrow[no head, from=2-3, to=1-2]
        \end{tikzcd}\]
    which is a homotopy from $f$ to $f$.
    Let us now consider three maps $f,g,h\colon X \to Y$ and let $\sigma_2$ be a homotopy between $f$ and $h$, $\sigma_3$ a homotopy between $f$ and $g$, and let $\sigma_0$ witness the degenerate $0$-face.
    Then we can picture the map $\tau_0 \colon \Lambda_1^3 \to \C$ induced by the tuple $(\sigma_0, \blank, \sigma_2,\sigma_3)$ with the following diagram:
    % https://q.uiver.app/?q=WzAsNCxbMCwyLCJYIl0sWzIsMiwiWSJdLFszLDEsIlkiXSxbMSwwLCJZIl0sWzEsMiwiIiwxLHsibGV2ZWwiOjIsInN0eWxlIjp7ImJvZHkiOnsibmFtZSI6ImRhc2hlZCJ9LCJoZWFkIjp7Im5hbWUiOiJub25lIn19fV0sWzAsMiwiXFxxdWFkIiwxLHsibGFiZWxfcG9zaXRpb24iOjYwLCJzdHlsZSI6eyJib2R5Ijp7Im5hbWUiOiJkYXNoZWQifX19XSxbMCwzLCJmIl0sWzMsMiwiIiwxLHsibGV2ZWwiOjIsInN0eWxlIjp7ImhlYWQiOnsibmFtZSI6Im5vbmUifX19XSxbMywxLCJnIiwyLHsibGFiZWxfcG9zaXRpb24iOjYwLCJsZXZlbCI6Miwic3R5bGUiOnsiaGVhZCI6eyJuYW1lIjoibm9uZSJ9fX1dLFswLDEsImgiLDIseyJzdHlsZSI6eyJib2R5Ijp7Im5hbWUiOiJkYXNoZWQifX19XV0=
    \[\begin{tikzcd}
            & Y \\
            &&& Y \\
            X && Y
            \arrow[Rightarrow, dashed, no head, from=3-3, to=2-4]
            \arrow["g", dashed, from=3-1, to=2-4]
            \arrow["f", from=3-1, to=1-2]
            \arrow[Rightarrow, no head, from=1-2, to=2-4]
            \arrow[Rightarrow, no head, crossing over, from=1-2, to=3-3]
            \arrow["h"', dashed, from=3-1, to=3-3]
        \end{tikzcd}\]
    where the dashed lines represent the boundary of the ``missing'' face.
    Now, since $\C$ is an $\infty$-category we can fill in this face and extend $\tau_0$ to a $3$-cell $\tau \colon \Delta^3 \to \C$.
    Observe now that the face $d_1(\tau)$ gives a homotopy between $g$ and $h$, so we have shown transitivity.
    \newline
    Finally setting $f = h$ in the above diagram shows homotopy is symmetric since $f$ is always homotopic to itself.
\end{proof}
\begin{proposition}[{\cite[\href{https://kerodon.net/tag/0040}{Proposition 0040}]{kerodon}}]
    Two maps $f,g \colon X \to Y$ in an $\infty$-category $\C$ are homotopic if and only if they are homotopic as morphisms in $\C^{op}$.
\end{proposition}
\begin{proof}
    One must show that demanding the existence of a $2$-cell $\sigma \in \C_2$ such that $d_0(\sigma) = \id_Y, d_1(\sigma)=g$ and $d_2(f)$ is equivalent to demanding the existence of a $2$-cell $\tau \in \C_2$ such that $d_0(\tau) = f, d_1(\tau) = g$ and $d_2(\tau) = \id_X$.
    \newline
    Let $f$ be homotopic to $g$.
    By the symmetry of the homotopy relation there exists a homotopy $\sigma$ from $g$ to $f$.
    Then the tuple $(\sigma, s_1(g), \blank, s_0(g))$ yields a map $\rho_0 \colon \Lambda_2^3 \to \C$ as follows:
    \[\begin{tikzcd}
            && X \\
            &&&& Y \\
            X &&& Y
            \arrow["g", dashed, from=3-1, to=2-5]
            \arrow["\id_X", dashed, from=3-1, to=1-3]
            \arrow["f", dashed,from=1-3, to=2-5]
            \arrow["\id_Y"', from=3-4, to=2-5]
            \arrow["f", crossing over,from=1-3, to=3-4]
            \arrow["g"', from=3-1, to=3-4]
        \end{tikzcd}\]
    where once again the dashed lines represent the boundary of the ``missing'' face from $\Lambda_2^3$.
    We fill in to get a $3$-cell $\rho \colon \Delta^3 \to \C$ with $d_2(\rho)=\tau$ as ``demanded'' above.
    The other direction is very similar and like most of these proofs it comes down to drawing the correct horn.
\end{proof}
%\TODO{Consider adding Corollary \cite[\href{https://kerodon.net/tag/00V0}{Corollary 00V0}]{kerodon}}
Now that we know what it means for morphisms of $\infty$-categories to be homotopic, we can define a composition of morphisms.
\begin{definition}[{\cite[\href{https://kerodon.net/tag/0042}{Definition 0042}]{kerodon}}]
    Let $\C$ be an $\infty$-category with morphisms
    \[\begin{tikzcd}
            X && Y \\
            & Z
            \arrow["f", from=1-1, to=1-3]
            \arrow["g", from=1-3, to=2-2]
            \arrow["h"', from=1-1, to=2-2]
        \end{tikzcd}\]
    We define $h$ to be a composition of $f$ and $g$ if there exists some $2$-simplex $\sigma \in \C$ such that $d_0(\sigma) = g, d_1(\sigma)=h$ and $d_2(\sigma)=f$.
    We say $\sigma$ witnesses $h$ as a composition of $f$ and $g$ and we will use the usual notation $h = g \circ f$.
\end{definition}
Observe that we have only defined composition up to homotopy.
We make this precise in the following proposition:
\begin{proposition}[{\cite[\href{https://kerodon.net/tag/0043}{Proposition 0043}]{kerodon}}]
    Let $\C$ be an $\infty$-category with morphisms $f$ and $g$ as follows:
    \[\begin{tikzcd}
            X && Y \\
            & Z
            \arrow["f", from=1-1, to=1-3]
            \arrow["g", from=1-3, to=2-2]
            \arrow["h"', dashed, from=1-1, to=2-2]
        \end{tikzcd}\]
    Then there exists a composition $h$ of $f$ and $g$ and any other morphism $X\to Z$ is a composition of $f$ and $g$ if and only if it is homotopic to $h$. \qed
\end{proposition}
Furthermore, compositions respect homotopy in the following sense:
\begin{proposition}[{\cite[\href{https://kerodon.net/tag/0048}{Proposition 0048}]{kerodon}}]
    Let $\C$ be an $\infty$-category with homotopic morphisms $f \sim f'\colon X \to Y$ and $g \sim g' \colon Y \to Z$.
    Let $h=g\circ f$ and $h' = g' \circ f'$.
    Then $h$ is homotopic to $h'$.\qed
\end{proposition}
\begin{remark}
    The nerve construction preserves compositions in the sense that for a $1$-category $\C$ with morphisms $f,g$ as above, there is a unique morphism $h\colon X\to Z$ in $\Nerve(\C)$ which is given by $f\circ g$ in $\C$.
\end{remark}
One can show that the nerve construction $\mathrm{Cat} \xrightarrow{\Nerve} \sset$ admits a left adjoint $\h$ and moreover that the counit of this adjunction is an isomorphism which in turn means that the nerve is fully faithful.
This can be shown directly and the interested reader can see for example \cite[Proposition 4.10.]{Rezk} or \cite[\href{https://kerodon.net/tag/002Y}{Subsection 002Y}]{kerodon} for proofs.
We now construct $\h$ directly, but only on $\infty$-categories, but we will delay the proof of the adjunction to chapter \ref{kanchap} to illustrate the usefulness of Kan extensions.
Analogously to the construction of the fundamental groupoid $\pi_{\leq 1}(X)$ of a topological space $X$, we can construct the homotopy category $\h\C$ of an $\infty$-category $\C$.
\begin{definition}\label{homhtpycat}
    Let $\C$ be an $\infty$-category.
    We denote by $\Hom_{\h\C}(X,Y)$ homotopy classes of morphisms $X\to Y \in \C$ and for a morphism $f \colon X \to Y$ in $\C$, we denote by $[f]$ its equivalence class in $\Hom_{\h\C}(X,Y)$.
\end{definition}
\begin{proposition}[{\cite[\href{https://kerodon.net/tag/004B}{Proposition 004B}]{kerodon}}]
    We have a unique composition of morphisms
    \[
        \circ \colon \Hom_{\h\C}(Y, Z) \times \Hom_{\h\C}(X, Y) \rightarrow \Hom_{\h\C}(X, Z)
    \]
    such that $[h] = [f] \circ [g]$ for any $h = f\circ g \in \C$.
    This composition law is both
    \begin{enumerate}
        \item associative in the sense that any triple $W\xrightarrow{f}X\xrightarrow{g}Y\xrightarrow{h}Z$ in $\C$ yields an equivalence
              \[
                  ([h] \circ [g])\circ [f] = [h] \circ ([g] \circ [f]) \in \Hom_{\h\C}(W,Z).
              \]
        \item unital in the sense that for any $X\in \C$ the homotopy class $[\id_X]$ of the identity on $X$ is a two-sided identity with respect to the composition law.
              In other words, for every $W\xrightarrow{f}X$ and every $X\xrightarrow{g}Y$ in $\C$, we have $[id_X] \circ [f] = [f]$ and $[g]\circ [id_X] = [g]$.
    \end{enumerate}
\end{proposition}
\begin{proof}
    The existence of the composition law follows directly from the previous two propositions.
    To prove $1.$ we pick compositions $u = g\circ f$, $v = h\circ g$ and $w = v\circ f$.
    This means that $([h]\circ [g])\circ [f] = [w]$ and $[h] \circ ([g]\circ [f]) = [h]\circ [u]$, so it remains to show that $w = h \circ u$.
    Choosing $2$-cells $\sigma_0, \sigma_2, \sigma_3$ witnessing the compositions $v = g\circ h, u = g\circ f$ and $w = v \circ f$, respectively, yields a map $\Lambda_1^3 \to \C$ as depicted in the following diagram:
    \[\begin{tikzcd}
            && X \\
            &&&& Z \\
            W &&& Y
            \arrow["w", dashed, from=3-1, to=2-5]
            \arrow["f", from=3-1, to=1-3]
            \arrow["v",from=1-3, to=2-5]
            \arrow["h"', dashed, from=3-4, to=2-5]
            \arrow["g", crossing over, from=1-3, to=3-4]
            \arrow["u"', dashed, from=3-1, to=3-4]
        \end{tikzcd}\]
    where the dashed lines represent the ``missing'' $2$-cell.
    Since $\C$ is an $\infty$-category we can extend this map to a $3$-cell $\Delta^3 \to \C$ essentially ``filling'' in the missing $2$-cell witnessing the desired composition $w = h \circ u$.
    \newline
    To prove $2.$ pick $X\in \C$ and maps $g \in \Hom_{\C}(X,Y)$ and $f \in \Hom_{\C}(W,X)$ and observe that the degenerate $2$-cells with boundaries as in the following diagrams:
    \[\begin{tikzcd}
            & X &&&& X \\
            \\
            X && Y && W && X
            \arrow["g"', from=3-1, to=3-3]
            \arrow["g", from=1-2, to=3-3]
            \arrow[no head, from=3-1, to=1-2]
            \arrow["{\id_X}"', shift right=1, no head, from=1-2, to=3-1]
            \arrow["f", from=3-5, to=1-6]
            \arrow["f"', from=3-5, to=3-7]
            \arrow[shift right=1, no head, from=1-6, to=3-7]
            \arrow["{\id_X}"', no head, from=3-7, to=1-6]
        \end{tikzcd}\]
    witness the compositions $g \circ \id_X = g$ and $\id_X \circ f = f$.
\end{proof}
We can now define the homotopy category $\h\C$ of an $\infty$-category $\C$.
\begin{definition}
    Let $\C$ be an $\infty$-category.
    Then we define $\h\C$ to be the $1$-category with objects of $\C$ as its objects and homotopy classes of morphisms as defined in \ref{homhtpycat} as its morphisms.
    The previous proposition provides identity morphisms $[\id_X]$ for any object $X\in \C$ and a composition law satisfying the axioms for being a $1$-category.
\end{definition}
\begin{example}\phantom{a}
    \begin{enumerate}
        \item $\h\Delta^n = [n] \simeq \{0 < 1 < \cdots < n\}$ and in general $\h\Nerve(\C)\cong \C$.
        \item For a topological space $X$ one can identify $\h\mathrm{Sing}(X)$ with $\pi_{\leq 1}(X)$.
    \end{enumerate}
\end{example}
\subsection{Isomorphims}
\begin{definition}
    Let $\C$ be an $\infty$-category and let $X\xrightarrow{f}Y$ in $\C$.
    We say $f$ is an isomorphism if $[f]$ is an isomorphism in $\h\C$.
    Isomorphims are also often called equivalences.
\end{definition}
\begin{example}
    Let $\C$ be a $1$-category.
    A morphism in $\C$ is an isomorphism if and only if it is an isomorphism in $\Nerve(\C)$.
\end{example}
\begin{definition}
    An $\infty$-groupoid is an $\infty$-category such that $\h\C$ is a groupoid, or in other words an $\infty$-category where every morphism is an isomorphism.
\end{definition}
\begin{example}
    Every Kan complex $K$ is an $\infty$-groupoid because every horn can be filled and filling the horns $\Lambda_0^2 \to K$ and $\Lambda_1^2 \to K$ yields inverses for any morphisms in $K$.
    %\newline
    %In particular, this means the singular complex $\mathrm{Sing}X$ of a topological space $X$ is an $\infty$-groupoid and one can show that $\h\mathrm{Sing}X$ is the fundamental groupoid of $X$.
\end{example}

As one should maybe expect, this works the other way around as well; $\infty$-groupoids are Kan complexes.
Thankfully, this is true, but it is a non-trivial and technical theorem which is the main focus of \cite{Joyal}.
For a proof, see \cite[Corollary 1.4]{Joyal} or \cite[Section 34]{Rezk}.
Inspired by \cite[Corollary 14.2.18.]{Groth} we can write the following commutative diagram of fully faithful functors:
\[\begin{tikzcd}
        {\mathrm{Grpd}} && {\mathrm{Cat}} \\
        \\
        {\mathrm{Kan}=\mathrm{Grpd_{\infty}}} && {\mathrm{Cat}_{\infty}} && \sset
        \arrow[from=1-1, to=1-3]
        \arrow["\Nerve"', from=1-1, to=3-1]
        \arrow[from=3-1, to=3-3]
        \arrow["\Nerve", from=1-3, to=3-3]
        \arrow[from=3-3, to=3-5]
        \arrow["\Nerve", from=1-3, to=3-5]
    \end{tikzcd}\]
\begin{definition}
    Let $\C$ be an $\infty$-category and let $D$ be a subcategory of $\h\C$, then we define the subcategory of $\C$ determined by $D$ as the pullback
    \[\begin{tikzcd}
            \D & \C \\
            {\Nerve(D)} & \h\C
            \arrow[from=1-1, to=1-2]
            \arrow[from=1-1, to=2-1]
            \arrow[from=2-1, to=2-2]
            \arrow[from=1-2, to=2-2]
            \arrow["\lrcorner"{anchor=center, pos=0.125}, draw=none, from=1-1, to=2-2]
        \end{tikzcd}\]
\end{definition}
\begin{definition}
    We will say that $\D$ is a full subcategory of $\C$ whenever $D$ is a full subcategory of $\h\C$.
\end{definition}
\begin{proposition}
    The subcategory $\D$ of an $\infty$-category $\C$ determined by a subcategory $D$ of $\h\C$ is an $\infty$-category.\qed
\end{proposition}
\begin{definition}
    The core of a $1$-category is the subcategory consisting of only the isomorphisms.
\end{definition}
\begin{definition}
    The core of an $\infty$-category $\C$ is the $\infty$-groupoid $\C^{\simeq}$ (also written $\C^{\text{core}}$ by some authors) obtained as the subcategory of $\C$ corresponding to the core of $\h\C$.
\end{definition}
We will say two objects in an $\infty$-category are isomorphic whenever there exists an isomorphism between them.
Furthermore being isomorphic is an equivalence relation on the objects of an $\infty$-category which means we can sensibly speak of isomorphism classes.


In addition to having a notion of isomorphism or equivalence of objects in an $\infty$-category we would like a notion of natural isomorphism or equivalence of $\infty$-categories, but first we define the $\infty$-category $\MAP(\C, \D)$ of functors between $\infty$-categories.
\subsection{Mapping spaces}
For ordinary $1$-categories $\C$ and $\D$ we can create the category $\Fun(\C, \D)$ with functors its objects and natural transformations as its morphisms.
We want to create an $\infty$-categorical analogue:
\begin{definition}
    Let $X, Y \in \sset$.
    We define $\MAP(X,Y)$ by $\MAP(X,Y)_n : = \Hom_{\sset}(\Delta^n \times X, Y)$.
\end{definition}
If $\sigma$ is some map $[m] \to [n]$ in $\Delta$, the induced map
\[
    \sigma^* : \MAP(X,Y)_n \to \MAP(X,Y)_m
\]
is defined by
\[
    (X \times \Delta^n \xrightarrow{f} Y) \mapsto (X \times \Delta^m \xrightarrow{\id_X \times \sigma}X \times \Delta^n \xrightarrow{f} Y).
\]
In particular, this means that $\MAP(X,Y)_0$ is precisely the set of maps between the simplicial sets $X$ and $Y$.
%There are many different notations in the literature for the mapping space or function complex as it is sometimes called.
%\cite{GoerssJardine}[Section 5] writes simply $\mathrm{Hom}(X,Y)$ and \cite{Rezk}[Section 15] writes $\Fun(X,Y)$.
Observe that $\MAP$ defines a functor $\sset^{op}\times \sset \to \sset$ and for each $n$ it is clear that we have a bijection between $\Hom(\Delta^n \times X, Y)$ and $\Hom(\Delta^n, \MAP(X,Y))$.
Furthermore, we can extend the bijection to any simplicial set:
\begin{proposition}[{\cite[Proposition 15.3.]{Rezk}}]
    Let $X,Y,Z \in \sset$, then there is a bijeciton
    \[
        \Hom(X\times Y, Z) \xrightarrow{\sim} \Hom(X, \MAP(Y,Z)). \pushQED{\qed}\qedhere\popQED
    \]
\end{proposition}
This proposition yields a natural isomorphism of simplicial sets $\MAP(X\times Y, Z) \cong \MAP(X, \MAP(Y,Z))$.


It can be shown that applying the same construction to $\infty$-categories $\C$ and $\D$ yields a new $\infty$-category $\MAP(\C, \D)$ with maps of simplicial sets as objects ($0$-cells) and natural transformations (maps $\C \times \Delta^1 \to \D$) as morphisms ($1$-cells).
Proving this is indeed an $\infty$-category uses machinery due to Joyal that I will not introduce in this text.
See for example \cite[Theorem 22.4.]{Rezk} or the proof of \cite[1.2.7.3]{HTT} on p.$94$.
This allows us to define an equivalence between $\infty$-categories:
\begin{definition}
    Let $\C$ and $\D$ be $\infty$-categories and $f,g \in \MAP(\C,\D)$ functors between them.
    We will say that a natural transformation $\varphi$ between $f$ and $g$ is a natural isomorphism  or a natural equivalence if it is an equivalence in $ \MAP(\C,\D)$.
\end{definition}
\begin{definition}
    Let $f \in \MAP(\C,\D)$ be a functor of $\infty$-categories.
    Then $f$ is a categorical equivalence if there exists a functor $g\in\MAP(\D,\C)$ and natural equivalences between $gf$ and $\id_{\C}$ and between $\id_{\D}$ and $fg$.
\end{definition}
\begin{proposition}[{\cite[Exercise 15.8.]{Rezk}}]
    Let $\C$ and $\D$ be ordinary $1$-categories.
    Then
    \[
        \Nerve(\Fun(\C, \D)) \simeq \MAP(\Nerve(\C), \Nerve(\D)).
    \]
\end{proposition}
\begin{proof}
    We will show they are the same on the level of $n$-cells for all $n$.
    First use $\Delta^n = \Nerve([n])$ and that the nerve preserves finite products to observe the following:
    \begin{align}
        \MAP(\Nerve(\C), \Nerve(\D))_n: & = \Hom_{\sset}(\Delta^n \times \Nerve(\C), \Nerve(\D))   \\
                                        & = \Hom_{\sset}(\Nerve([n])\times \Nerve(\C), \Nerve(\D)) \\
                                        & = \Hom_{\sset}(\Nerve([n]\times \C), \Nerve(\D))
    \end{align}
    Finally, we use fully-faithfulness of the nerve to get
    \begin{align}
        \Hom_{\sset}(\Nerve([n]\times \C), \Nerve(\D)) & \cong \Hom_{\mathrm{Cat}}([n] \times \C, \D) \\
                                                       & \cong \Hom_{\mathrm{Cat}}([n], \Fun(\C, \D)) \\
                                                       & = (\Nerve(\Fun(\C, \D)))_n.
    \end{align}
    To complete the proof, we note that these isomorphisms are all natural in $n$.
\end{proof}
\begin{definition}
    Let $\C$ be an $\infty$-category with objects $X$ and $Y$.
    Then we define the mapping space $\map_{\C}(X,Y)$ as the pullback
    \[\begin{tikzcd}
            {\map_{\C}(X,Y)} && {\Fun(\Delta^1,\C)} \\
            \\
            {\{(X,Y)\}} && \C\times\C
            \arrow[from=1-1, to=1-3]
            \arrow[from=1-3, to=3-3]
            \arrow[from=3-1, to=3-3]
            \arrow[from=1-1, to=3-1]
            \arrow["\lrcorner"{anchor=center, pos=0.125}, draw=none, from=1-1, to=3-3]
        \end{tikzcd}\]
    which, in fact, is an actual Kan complex\footnote{It is named mapping space because of the convention among many authors to call Kan complexes and $\infty$-groupoids spaces.}.
\end{definition}
For a proof that this is always a Kan complex, see \cite[Proposition 45.2]{Rezk}.
It is worth mentioning that we only define the mapping space up to homotopy so there are other useful ways to model it that we are not mentioning here.
\begin{definition}
    We say a functor $f \in \MAP(\C,\D)$ of $\infty$-categories is essentially surjective if the functor ${\h f \in \Fun(\h\C, \h\D)}$ is essentially surjective.
\end{definition}
\begin{definition}
    We say a functor $f \in \MAP(\C,\D)$ of $\infty$-categories is fully faithful if the functor ${\map_{\C}(X,Y) \to \map_{\D}(f(X),f(Y))}$ is an equivalence for all $X,Y \in \C$.
\end{definition}
Charles Rezk names the following the ``fundamental theorem'' of $\infty$-categories (and he names the corresponding result for $1$-categories the ``fundamental theorem of category theory'' though he mentions Yoneda's lemma might be more deserving of that name).
\begin{theorem}[{\cite[Theorem 48.2.]{Rezk}}]
    A functor $f \colon \C \to \D$ of $\infty$-categories is a categorical equivalence if and only if it is fully faithful and essentially surjective. \qed
\end{theorem}
We can also define a weaker notion of equivalence between simplicial sets.
\begin{definition}
    We say a map $f \colon X \to Y$ of simplicial sets is a weak homotopy equivalence if for any space $Z$, the induced functor
    \[
        \MAP(f, Z) \colon \MAP(Y, Z) \to \MAP(X,Z)
    \]
    is a categorical equivalence.
\end{definition}
An immmediate observation is that any categorical equivalence between $\infty$-categories $X$ and $Y$ is also a weak homotopy equivalence, but the converse is not necessarily true:
\begin{example}
    $\Delta^0 \hookrightarrow \Delta^1$ is a weak homotopy equivalence but not a categorical equivalence.
\end{example}
\subsection{Adjunctions}
We give an $\infty$-categorical definition of adjoint functors.
\begin{definition}[{\cite[\href{https://kerodon.net/tag/02EK}{Definition 02EK}]{kerodon}}]
    Let $F \colon \C \to \D$ and $G \colon \D \to \C$ be functors of $\infty$-categories.
    We say that a pair of natural transformations $\eta \colon \id_{\C} \to G\circ F$ and $\varepsilon \colon F\circ G \to \id_{\D}$ are compatible up to homotopy if the following conditions are satisfied:
    \begin{enumerate}
        \item The identity isomorphism $\id_F \colon F \to F$ is a composition of natural transformations:
              \[
                  F = F \circ \id_{\C} \xrightarrow{\id_F \circ \eta} F \circ G \circ F \xrightarrow{\varepsilon \circ \id_F} \id_{\D} \circ F = F.
              \]
        \item The identity isomorphism $\id_G \colon G \to G$ is a composition of natural transformations:
              \[
                  G = \id_{\D} \circ G \xrightarrow{\eta \circ \id_G} G \circ F \circ G \xrightarrow{\id_G \circ \varepsilon} G \circ \id_{\D} = G.
              \]
    \end{enumerate}
\end{definition}
\begin{definition}
    We say that a natural transformation $\eta \colon \id_{\C} \to G\circ F$ is the unit of an adjunction if there exists a natural transformation $\varepsilon \colon F\circ G \to \id_{\D}$ which is compatible with $\eta$ up to homotopy.
    \newline
    We say that a natural transformation $\varepsilon \colon F\circ G \to \id_{\D}$ is the counit of an adjunction if there exists a natural transformation $\eta \colon \id_{\C} \to G\circ F$ which is compatible with $\varepsilon$ up to homotopy.
\end{definition}
\begin{definition}
    Let $F \colon \C \to \D$ and $G \colon \D \to \C$ be functors of $\infty$-categories.
    We say that $F$ is a left adjoint to $G$, and $G$ a right adjoint to $F$, if there exists a natural transformation $\eta \colon \id_{\C} \to G\circ F$ which is the unit of an adjunction between $F$ and $G$.
\end{definition}
\begin{example}
    An equivalence between $\infty$-categories is both a left and a right adjoint.
\end{example}
\begin{example}
    A composition of two left/right adjoints is again a left/right adjoint.
\end{example}
It can be shown that this notion of adjunctions is compatible with the $1$-categorical notion of adjunctions in the sense that a natural transformation $\eta \colon \id_{\C} \to G\circ F$ is the unit of a $1$-categorical adjunction if and only if its nerve is the unit of an $\infty$-categorical adjunction.
For $\C$ and $\D$ $1$-categories one can equivalently define an adjunction as a pair of functors $L \colon \C \to \D$ and $R\colon \D \to \C$ such that there is a bijection $\Hom_{\D}(L(\blank), \blank) \simeq \Hom_{\C}(\blank, R(\blank))$.
Similarly, it can be shown that a pair of functors between $\infty$-categories is an adjunction if and only if we have a bijection of mapping spaces:
\begin{proposition}[{\cite[\href{https://kerodon.net/tag/02FX}{Corollary 02FX}]{kerodon}}]\label{adjUnivProp}
    Let $F \colon \C \to \D$ and $G \colon \D \to \C$ be functors of $\infty$-categories and let $\eta : \id_{\C} \to G\circ F$ be a natural transformation.
    Then $\eta$ is the unit of an adjunction between $F$ and $G$ if and only if, for any pair of objects $X \in \C, Y \in \D$, the following composition is an isomorphism of $\infty$-groupoids:
    \[
        \map_{\D}(F(X), Y) \xrightarrow{G} \map_{\C}((G\circ F)(X), G(Y)) \xrightarrow{\circ [\eta_X]} \map_{\C}(X, G(Y)).
    \]
\end{proposition}
\end{document}