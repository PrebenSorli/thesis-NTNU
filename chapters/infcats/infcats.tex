\documentclass[../../thesis.tex]{subfiles}

\begin{document}
%\TODO{Speak about $(\infty, n)$-categories.}
What Lurie \cite{HTT} calls $\infty$-categories were originally called restricted Kan complexes by Boardman and Vogt \cite{BoardmanVogt}, but without intent of using them for $\infty$-categories.
The first development of such a theory was done by Joyal in \cite{Joyal} who called them quasicategories.
As most of this thesis follows Lurie's works very closely, we will follow his convention and use the name $\infty$-categories.
While \cite{HTT} gives a good introduction to $\infty$-categories extending on the work of Joyal, his web-project \cite{kerodon}\footnote{This is kind of his solo-version of the Stacks project for $\infty$-categories.} reworks a lot of the foundations and we take a lot of inspiration from this presentation.
\section{Simplicial sets}
Originally, simplicial sets was used to rephrase the homotopy theory of spaces in combinatorial terms.
There are many good introductions to simplicial sets, depending on what you want to use them for, but Friedman's \cite{friedman2021elementary} was enlightening for the author of this thesis.
For algebraic topologists, Peter May's \cite{MAY} is a good introduction to semi-simplicial topology.\footnote{\prepp{Is this even relevant?}}
\begin{definition}
    Usually denoted by $\Delta$, the simplex category or the simplicial category is the category with linearly ordered sets $[n]=\{0,1,2,\cdots ,n\}$ as its objects and order-preserving maps between them as its morphisms.
    That is, for a map $\varphi:[m] \to [n]$ we have that $0 \leq \varphi(i)\leq \varphi(j)\leq n$ for each $0 \leq i \leq j \leq m$.
\end{definition}
We denote by $\delta^i$ the elementary face operator $[n-1] \to [n]$ and by $\sigma^i$ the elementary degeneracy operator $[n+1] \to [n]$ given by
\[
    \begin{aligned}
        \delta^i(j) =
        \begin{cases}
            j   & \text{ if } j < i    \\
            j+1 & \text{ if } j \geq i
        \end{cases}, \quad
        \sigma^i(j) =
        \begin{cases}
            j   & \text{ if } j \leq i \\
            j-1 & \text{ if } j > i
        \end{cases}
    \end{aligned}
\]
\begin{remark}
    All morphisms in $\Delta$ are compositions of such morphisms.
\end{remark}
\begin{definition}
    We define the category $\sset$, also denoted $\SET_{\Delta}$ by Lurie, of simplicial sets as $\SET$-valued presheaves on $\Delta$, i.e. functors $\Delta^{op}\to\SET$.
\end{definition}
Let $X \in \sset : = \Fun(\Delta^{op}, \SET)$. We wil denote by $X_n$ the set $X([n])$ of $n$-simplices or also called $n$-cells of $X$.
We define the standard $n$-simplex as $\Delta^n:=y([n])$ where $y$ is the Yoneda embedding, meaning $\Delta^n$ is the presheaf $\Hom_{\Delta}(\blank,[n])$.
By the Yoneda lemma $\Hom_{\Fun(\Delta^{op}, \SET)}(\Delta^n, X) \simeq X_n$, so we can identify each simplex $x\in X_n$ with a map $x:\Delta^n\to X$.
This application of the Yoneda lemma is a crucial part of the theory of simplicial sets and we will more often than not consider $n$-simplices of a simplicial set $X$ as maps of simplicial sets instead.
Observe, moreover that composition with the elementary face operator gives us a map $\Delta^{n-1}\to\Delta^n$.
%by $\delta^i \circ\blank$.
\begin{definition}
    For a simplicial set $X$, we define the face and degeneracy maps
    \[
        d_i: = X(\delta^i): X_{n} \to X_{n-1}, \quad s_i: = X(\sigma^i): X_{n} \to X_{n+1}
    \]
    where both maps are given by composition with $\delta^i$ and $\sigma^i$ respectively.
\end{definition}
\begin{example}
    The standard $0$-simplex $\Delta^0:=\Hom(\blank, [0])$ is a terminal object in $\sset$, meaning it maps any $[m]\in \Delta$ to a singleton.
    This is usually just referred to as the point and denoted $*$.
\end{example}
\begin{example}[{\cite[\href{https://kerodon.net/tag/000P}{Remark 000P}]{kerodon}}]
    Let $X \in \sset$ and suppose we have subsets $T_n\subseteq X_n$ for every $n\geq 0$ such that $d_i(T_n)\subseteq T_{n-1}$ and $s_i(T_n) \subseteq T_{n+1}$.
    Then the collection $\{T_n\}_{n\geq 0}$ is a simplicial set we will call a simplicial subset $T\subseteq X$.
\end{example}
\begin{definition}
    We define the boundary $\partial\Delta^n$ of $\Delta^n$ as the simplicial set
    \[
        (\partial\Delta^n)_m = (\partial\Delta^n)([m]) :=\{\alpha \in \Hom_{\Delta}([m], [n]) | [n] \not\subseteq \mathrm{im}(\alpha)\}.
    \]
    Observe that $\partial\Delta^0=\emptyset$ because every map $[m] \to [0]$ is surjective.
\end{definition}
\begin{definition}
    For $0\leq i \leq n$, we define the horn $\Lambda_i^n$ as the simplicial set
    \[
        (\Lambda_i^n)_m =(\Lambda_i^n)([m]) :=\{\alpha \in \Hom_{\Delta}([m], [n]) | \delta^i[n] \not\subseteq \mathrm{im}(\alpha)\}.
    \]
\end{definition}
Observe that the horn is inside the boundary.
We usually refer to $\Lambda_i^n$ as the $i$th horn in $\Delta^n$ and we will call the horns such that $0<i<n$ the inner horns.
\begin{example}
    \TODO{Write about $\mathrm{Sing}(X)$ for a topological space $X$.}
\end{example}
\begin{example}\label{NerveDef}
    We define the nerve $\Nerve(\C)$ of a $1$-category $\C$ by
    \[
        \Nerve(\C) : = \Hom_{\mathrm{Cat}}([\blank], \C)
    \]
    where we view the sets $[n]$ as categories (posets with a map $i$ to $j$ whenever $i\leq j$).
    Observe that for any order-preserving morphism $\alpha : [m] \to [n]$ we get a map
    \[
        \Hom_{\mathrm{Cat}}([n], \C) \xrightarrow{\blank \circ \alpha} \Hom_{\mathrm{Cat}}([m], \C)
    \]
    and it is clear that the nerve is a simplicial set with $\Nerve(\C)_n = \Hom_{\mathrm{Cat}}([n], \C)$.
    \newline
    Observe furthermore that for a functor $F: \C \to \D$ we get a simplicial map $\Nerve(F):\Nerve(\C) \to \Nerve(\D)$ by sending $n$-cells $\varphi : [n] \to \C$ in $\Nerve(\C)_n$ to $n$-cells $F(\varphi) : [n] \to \D$ in $\Nerve(\D)_n$, so the construction is functorial.
    It should also be clear that the set of objects of $\C$ is identified with the $0$-cells $\Nerve(\C)_0$ and the morphisms with the $1$-cells $\Nerve(\C)_1$.
    Additionally, the $2$-cells $\Nerve(\C)_2$ is in bijective with the set of composable pairs of morphisms in $\C$.
    We will talk more about composition of morphisms in the next section.
    It is possible to give a general description of the nerve's $n$-cells, see for example \cite[Proposition 4.4.]{Rezk}, but the interesting stuff is really in the dimensions $\{0,1,2\}$.
\end{example}
\section{$\infty$-categories}
Before we give a precise definition, we will take a closer look at the nerve construction.
Clearly, we want the nerve of a $1$-category to give us an $\infty$-category and most of this thesis will revolve around nerves of certain poset-categories of topological spaces.
Nerves of categories are not just any ordinary simplicial sets, but simplicial sets with some more structure inherited from the underlying $1$-category.
For instance, $1$-categories have composition of morphisms.
Take for example
\[\begin{tikzcd}
        X && Y \\
        & Z
        \arrow["f", from=1-1, to=1-3]
        \arrow["g", from=1-3, to=2-2]
        %\arrow["h"', from=1-1, to=2-2]
    \end{tikzcd}\]
in some ordinary $1$-category $\C$.
This diagram gives us a morphism $\Lambda_1^2\to \Nerve(\C)$ of simplicial sets, but in $\C$ $f$ and $g$ can be composed to a morphism $h: X \to Z$ which in turn gives a unique way to extend the simplicial map $\Lambda_1^2\to \Nerve(\C)$ to a map $\Delta^2 \to \Nerve(\C)$.
If we instead look at the outer horns $\Lambda_0^2$ and $\Lambda_2^2$ we will not necessarily have a way to extend morphisms to $\Delta^2$ in general.
For example the diagram
\[\begin{tikzcd}
        X && Y \\
        & X
        \arrow["g", from=1-3, to=2-2]
        \arrow["\id_X"', from=1-1, to=2-2]
    \end{tikzcd}\]
gives a map $\Lambda_2^2\to \Nerve(\C)$, but extending this to a morphism $\Delta^2 \to \Nerve(\C)$ would amount to finding a right-inverse to $g$, which of course is not something we can always do in general, unless $\C$ was a groupoid.
This property of extending a morphism from a horn to the standard $n$-simplex is sometimes also called filling the horn, and we will see that it is a defininig property for $\infty$-categories.
In fact, the existence of horn fillings completely classifies the simplicial sets which are nerves of categories:
\begin{proposition}[{\cite[Proposition 1.1.2.2]{HTT}}]
    Let $X \in \sset$.
    Then the following conditions are equivalent:
    \begin{enumerate}
        \item There exists a small category $\C$ with an isomorphism $X \simeq \Nerve(\C)$.
        \item Every inner horn $\Lambda_i^n \to X$ of $X$ can be filled in an unique way.
              Or, in other words, for any solid diagram as below, there is a unique dotted arrow making it commute:
              \[\begin{tikzcd}
                      {\Lambda_i^n} && X \\
                      \\
                      {\Delta^n}
                      \arrow[from=1-1, to=1-3]
                      \arrow[from=1-1, to=3-1]
                      \arrow["{\exists!}"', dashed, from=3-1, to=1-3]
                  \end{tikzcd}\]
    \end{enumerate}
\end{proposition}
Simplicial sets which admits extensions for all horn inclusions are called Kan complexes:
\begin{definition}
    A simplicial set $X$ is a Kan complex if it satisfies the following condition:
    For $0 \leq i \leq n$, any map $\sigma_0 : \Lambda_i^n \to X$ can be extended to a map $\sigma: \Delta^n \to X$.
\end{definition}
\begin{proposition}
    Groupoids are Kan complexes.
\end{proposition}
\begin{proof}
    All morphisms are invertible, so all horns can be filled.
\end{proof}
As we saw in the example of a map $\Lambda_2^2\to \Nerve(\C)$ above, whenever there's non-invertible morphisms around some outer horns will be impossible to fill.
This motivates the definition of an $\infty$-category.
The following definition is due to Boardman and Vogt \cite{BoardmanVogt} who defined weak Kan complexes as simplicial sets satisfying what they called the restricted Kan condition:
\footnote{Maybe more commonly known as the weak Kan extension condition. Expand on this with a historical account.}:
\begin{definition}[Boardman and Vogt \cite{BoardmanVogt}]
    A simplicial set $X$ is an $\infty$-category if it satisfies the following condition:
    For $0 < i < n$, any map $\sigma_0 : \Lambda_i^n \to X$ can be extended to a map $\sigma: \Delta^n \to X$.
\end{definition}

This means that any Kan complex is an $\infty$-category, and in particular so is $\mathrm{Sing}(X)$ for a topological space $X$.
Additionally, observe that the nerve $\Nerve(\C)$ of an ordinary category $\C$ is an $\infty$-category.
Because the nerve functor is fully faithful (see example \ref{NerveKan}), many authors choose to omit its notation altogether.
\prepp{Consider if we should omit it as well. Currently we have been using it, but it is fucking annoying to remember writing $\Nerve$ all the time.}
In many ways $\infty$-categories behave similarly to ordinary categories and we will often write about them almost as if they were ordinary categories instead.
For example, we will use the terminology of ordinary category theory and refer to the vertices and edges of our simplicial sets as objects and morphisms in our $\infty$-categories.
There are however some obvious differences between ordinary categories and $\infty$-categories which needs adressing before adopting complete $1$-categorical language.
For example, and perhaps most crucially, we have higher-level maps given by simplices of dimension $n\geq 2$.
\TODO{Write about $\MAP$.}
While we have seen that nerves of categories admit unique horn extensions, this condition is dropped for general $\infty$-categories and hence, composition of morphisms in an $\infty$-category are not necessarily unique, but rather unique up to homotopy.
Before we can make this precise, we must define what we mean by homotopy.
\subsection{Homotopy}
\begin{definition}[{\cite[\href{https://kerodon.net/tag/003V}{Definition 003V}]{kerodon}}]
    Let $\C$ be an $\infty$-category and $f,g \in \Hom_{\C}(X,Y)$.
    We define a homotopy between $f$ and $g$ as a $2$-simplex $\sigma \in \C$ with boundary specified by $d_0(\sigma) = \id_Y, d_1(\sigma) = g$ and $d_2(\sigma)=f$ as illustrated in the diagram
    \[\begin{tikzcd}
            X && Y \\
            & Y
            \arrow["f", from=1-1, to=1-3]
            \arrow["g"', from=1-1, to=2-2]
            \arrow["{\id_Y}", from=1-3, to=2-2].
        \end{tikzcd}\]
    We say $f$ and $g$ are homotopic if such a homotopy $\sigma$ exists.
\end{definition}
\begin{example}
    For a $1$-category $\C$ two morphisms $f,g \in \Hom_{\C}(X,Y)$ are homotopic in $\Nerve(\C)$ if and only if $f=g$.
\end{example}
\begin{proposition}[{\cite[\href{https://kerodon.net/tag/003Z}{Proposition 003Z}]{kerodon}}]
    Let $\C$ be an $\infty$-category and $X,Y$ objects of $\C$.
    Then homotopy is an equivalence relation on $\Hom_{\C}(X,Y)$.
\end{proposition}
\begin{proof}
    \TODO{Prove. Both Rune and Lurie have nice proofs.}
\end{proof}
\begin{proposition}
    $f,g \in \Hom_{\C}(X,Y)$ are homotopic if and only if they are homotopic as morphisms in $\C^{op}$.
\end{proposition}
\begin{proof}
    \TODO{Lurie.}
\end{proof}
\TODO{Consider adding Corollary \cite[\href{https://kerodon.net/tag/00V0}{Corollary 00V0}]{kerodon}}
Now that we know what it means for morphisms of $\infty$-categories to be homotopic, we can define a composition of morphisms.
\begin{definition}[{\cite[\href{https://kerodon.net/tag/0042}{Definition 0042}]{kerodon}}]
    Let $\C$ be an $\infty$-category with morphisms
    \[\begin{tikzcd}
            X && Y \\
            & Z
            \arrow["f", from=1-1, to=1-3]
            \arrow["g", from=1-3, to=2-2]
            \arrow["h"', from=1-1, to=2-2]
        \end{tikzcd}\]
    We define $h$ to be a composition of $f$ and $g$ if there exists some $2$-simplex $\sigma \in \C$ such that $d_0(\sigma) = g, d_1(\sigma)=h$ and $d_2(\sigma)=f$.
    We say $\sigma$ witnesses $h$ as a composition of $f$ and $g$ and we will use the usual notation $h = g \circ f$.
\end{definition}
Observe that we have only defined composition up to homotopy.
We make this precise in the following proposition:
\begin{proposition}[{\cite[\href{https://kerodon.net/tag/0043}{Proposition 0043}]{kerodon}}]
    Let $\C$ be an $\infty$-category with morphisms $f$ and $g$ as follows:
    \[\begin{tikzcd}
            X && Y \\
            & Z
            \arrow["f", from=1-1, to=1-3]
            \arrow["g", from=1-3, to=2-2]
            \arrow["h"', dashed, from=1-1, to=2-2]
        \end{tikzcd}\]
    Then there exists a composition $h$ of $f$ and $g$ and any other morphism $X\to Z$ is a composition of $f$ and $g$ if and only if it is homotopic to $h$.
\end{proposition}
\prepp{The proof seems kind of technical, so might not be worth doing. Might do it if there's time or Rune thinks I should.
    Should consider adding \cite[\href{https://kerodon.net/tag/000Z}{Exercise 000Z}]{kerodon} as this exercise is used in a lot of the proofs in this subsection.}
Furthermore, compositions respect homotopy in the following sense:
\begin{proposition}[{\cite[\href{https://kerodon.net/tag/0048}{Proposition 0048}]{kerodon}}]
    Let $\C$ be an $\infty$-category with homotopic morphisms $f \sim f':X \to Y$ and $g \sim g': Y \to Z$.
    Let $h=f\circ g$ and $h' = f' \circ g'$.
    Then $h$ is homotopic to $h'$.
\end{proposition}
\begin{remark}
    The nerve construction preserves compositions in the sense that for a $1$-category $\C$ with morphisms $f,g$ as above, there is a unique morphism $h:\Nerve(X)\to \Nerve(Z)$ in $\Nerve(\C)$ which is given by $f\circ g$ in $\C$.
\end{remark}
One can show that the nerve construction $\mathrm{Cat} \xrightarrow{\Nerve} \sset$ admits a left adjoint $\h$ and moreover that the counit of this adjunction is an isomorphism which in turn means that the nerve is fully faithful.
This can be shown directly and the interested reader can see for example \cite[Proposition 4.10.]{Rezk} or \cite[\href{https://kerodon.net/tag/002Y}{Subsection 002Y}]{kerodon} for proofs, but we will instead delay the proof to section \ref{KanSection} to illustrate the usefulness of Kan extensions.
We now construct this left adjoint directly, but delay the proof of the adjunction.
Analogously to the construction of the fundamental groupoid $\pi_{\leq 1}(X)$ of a topological space $X$, we can construct the homotopy category $\h\C$ of an $\infty$-category $\C$.
\prepp{Writing about the fundamental gruopoid could be smart.}
\begin{definition}\label{homhtpycat}
    Let $\C$ be an $\infty$-category.
    We denote by $\Hom_{\h\C}(X,Y)$ homotopy classes of morphisms $X\to Y \in \C$ and for a morphism $f \in \Hom_{\C}(X,Y)$, we denote by $[f]$ its equivalence class in $\Hom_{\h\C}(X,Y)$.
\end{definition}
\begin{proposition}[{\cite[\href{https://kerodon.net/tag/004B}{Proposition 004B}]{kerodon}}]
    We have a unique composition of morphisms
    \[
        \circ : \Hom_{\h\C}(Y, Z) \times \Hom_{\h\C}(X, Y) \rightarrow \Hom_{\h\C}(X, Z)
    \]
    such that $[h] = [f] \circ [g]$ for any $h = f\circ g \in \C$.
    This composition law is both
    \begin{enumerate}
        \item associative in the sense that any triple $W\xrightarrow{f}X\xrightarrow{g}Y\xrightarrow{h}Z$ in $\C$ yields an equivalence
              \[
                  ([h] \circ [g])\circ [f] = [h] \circ (g \circ f) \in \Hom_{\h\C}(W,Z).
              \]
        \item unital in the sense that for any $X\in \C$ the homotopy class $[\id_X]$ of the identity on $X$ is a two-sided identity with respect to the composition law.
              In other words, for every $W\xrightarrow{f}X$ and every $X\xrightarrow{g}Y$ in $\C$, we have $[id_X] \circ [f] = [f]$ and $g\circ [id_X] = [g]$.
              This means that $([h]\circ [g])\circ [f] = [w]$ and $[h] \circ ([g]\circ [f]) = [h]\circ [u]$, so it remains to show that $[w] = [h] \circ [u]$.
    \end{enumerate}
\end{proposition}
\begin{proof}
    The existence of the composition law follows directly from the previous two propositions.
    To prove $1.$ we pick compositions $u = g\circ f$, $v = h\circ g$ and $w = v\circ f$.
    Choosing $2$-cells $\sigma_0, \sigma_2, \sigma_3$ witnessing the compositions $v = g\circ h, u = g\circ f$ and $w = v \circ f$, respectively yields a map $\Lambda_1^3 \to \C$ as depicted in the following diagram:
    \[\begin{tikzcd}
            && X \\
            &&&& Z \\
            W &&& Y
            \arrow["w", dotted, from=3-1, to=2-5]
            \arrow["f", from=3-1, to=1-3]
            \arrow["v",from=1-3, to=2-5]
            \arrow["h"', dotted, from=3-4, to=2-5]
            \arrow["g", from=1-3, to=3-4]
            \arrow["u"', dotted, from=3-1, to=3-4]
        \end{tikzcd}\]
    where the dotted lines represent the ``missing'' $2$-cell.
    Since $\C$ is an $\infty$-category we can extend this map to a $3$-cell $\Delta^3 \to \C$ essentially ``filling'' in the missing $2$-cell witnessing the desired composition $w = h \circ u$.
    \newline
    To prove $2.$ pick $X\in \C$ and maps $g \in \Hom_{\C}(X,Y)$ and $f \in \Hom_{\C}(W,X)$ and observe that the degenerate $2$-cells with boundaries as in the following diagrams:
    \[\begin{tikzcd}
            & X &&&& X \\
            \\
            X && Y && W && X
            \arrow["g"', from=3-1, to=3-3]
            \arrow["g", from=1-2, to=3-3]
            \arrow[no head, from=3-1, to=1-2]
            \arrow["{\id_X}"', shift right=1, no head, from=1-2, to=3-1]
            \arrow["f", from=3-5, to=1-6]
            \arrow["f"', from=3-5, to=3-7]
            \arrow[shift right=1, no head, from=1-6, to=3-7]
            \arrow["{\id_X}"', no head, from=3-7, to=1-6]
        \end{tikzcd}\]
    witnesses the compositions $g \circ \id_X = g$ and $\id_X \circ f = f$.
\end{proof}
We can now define the homotopy category $\h\C$ of an $\infty$-category $\C$.
\begin{definition}
    Let $\C$ be an $\infty$-category.
    Then $\h\C$ is the $1$-category with objects of $\C$ as its objects and homotopy classes of morphisms as defined in \ref{homhtpycat} as its morphisms.
    The previous proposition provides identity morphisms $[\id_X]$ for any object $X\in \C$ and composition law satisfying the axioms for being a $1$-category.
\end{definition}
\end{document}