\documentclass[../thesis.tex]{subfiles}

\begin{document}
In 1965 Jean-Louis Verdier introduced Verdier duality for locally compact topological spaces, thus generalizing the classical theory of Poincaré duality for manifolds.
Verdier Duality is a cohomological duality allowing for exchanging cohomology for cohomology with compact support.
More precisely it states that the derived functor of the compactly supported direct image functor has a right adjoint in the derived category of sheaves.
By using sheaf cohomology one can derive the classical Poincaré duality as a special case.
In his book ``Higher Algebra'' Jacob Lurie extends the theory to the $\infty$-categorical setting by showing there is an equivalence between sheaves and cosheaves valued in $\infty$-categories.
This thesis follows this proof closely, expanding and adding details where necesarry.
To introduce the relevant background on sheaves and $\K$-sheaves valued in stable $\infty$-categories we introduce and utilize Kan extensions, an ubiquitous concept in category theory.
\end{document}