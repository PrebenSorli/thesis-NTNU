\documentclass[../../thesis.tex]{subfiles}
\begin{document}
\TODO{Write some general stuff.}
\TODO{Where should I write about adjunctions?}
\TODO{Write about joins and slices before defining limits and colimits.}
\section{Joins and slices}
In this section we will introduce the join and slice constructions.
We will start with a recollection of what these constructions are in the case of ordinary $1$-categories before defining the appropriate $\infty$-categorical notions.
For most people, at least for me, the slice construction is very familiar while the join maybe not so much.
These two constructions will ultimately give us a way to talk about the right notions of limits and colimits in the world of $\infty$-categories.
\begin{definition}\label{sliceobj}
    Let $\C$ be a $1$-category and $C\in \C$.
    The slice category, or over category, $\C_{/C}$ is the category with arrows $C'\to C$ in $\C$ as objects and commutative triangles in $\C$ as its morphisms.
    The coslice category, or under category, $\C_{C/}$ is the category with arrows $C\to C'$ in $\C$ as objects and commutative triangles in $\C$ as its morphisms.
\end{definition}
\begin{remark}
    We have pullbacks
    \[\begin{tikzcd}
            {\C_{C/}} && {\Fun([1],\C)} && {\C_{/C}} \\
            \\
            {\{C\}} && \C && {\{C\}}
            \arrow[from=3-1, to=3-3]
            \arrow[from=1-1, to=3-1]
            \arrow[from=1-1, to=1-3]
            \arrow["{\mathrm{ev}_0}"', from=1-3, to=3-3]
            \arrow[from=1-5, to=1-3]
            \arrow[from=1-5, to=3-5]
            \arrow[from=3-5, to=3-3]
            \arrow["{\mathrm{ev}_1}", from=1-3, to=3-3]
            \arrow["\lrcorner"{anchor=center, pos=0.125, rotate=-90}, draw=none, from=1-5, to=3-3]
            \arrow["\lrcorner"{anchor=center, pos=0.125}, draw=none, from=1-1, to=3-3]
        \end{tikzcd}\]
    where $\mathrm{ev}_0:\Fun([1],\C) \to \Fun(\{0\}, \C) \simeq \C$ and $\mathrm{ev}_1:\Fun([1],\C) \to \Fun(\{1\}, \C) \simeq \C$.
\end{remark}
The above remark says that we can identify the slice and coslice categories with fibers of the evaluation functors $\mathrm{ev}_0$ and $\mathrm{ev}_1$ and we will use this idea to define the notion of slicing over (and under) diagrams.
\begin{definition}\label{slicefun}
    Let $\C$ and $\D$ be $1$-categories.
    For each $C\in \C$, we let $\underline{C} : \D \to \C$ denote the constant functor sending each $D\in \D$ to $C$ and each morphism to $\id_C$.
    For each functor $F:\D\to\C$ we denote by $\C_{/F}$ the fiber product
    \[\begin{tikzcd}
            {\C_{/F}} && {\Fun(\D,\C)_{/F}} \\
            \\
            \C && {\Fun(\D,\C)}
            \arrow[from=1-1, to=1-3]
            \arrow[from=1-1, to=3-1]
            \arrow[from=3-1, to=3-3]
            \arrow[from=1-3, to=3-3]
            \arrow["\lrcorner"{anchor=center, pos=0.125}, draw=none, from=1-1, to=3-3]
        \end{tikzcd}\]
    where the bottom arrow is given by $C \mapsto \underline{C}$.
    Dually, we denote by $\C_{F/}$ the fiber product ${\C \times_{\Fun(\D,\C)}\Fun(\D,\C)_{F/}}$.
    Here $\Fun(\D,\C)_{/F}$ and $\Fun(\D,\C)_{F/}$ are simply the slice and coslice categories of definition \ref{sliceobj}.
\end{definition}
\begin{definition}
    Let $\C$ and $\D$ be $1$-categories.
    We define the join $\C \star \D$ of $\C$ and $\D$ as the category with $\C \coprod \D$ as its object and for objects $X, Y$ morphisms given by:
    \[
        \Hom_{\C \star \D}(X,Y) : =
        \begin{cases}
            \Hom_{\C}(X,Y) & \text{ if } X,Y \in \C,         \\
            \Hom_{\D}(X,Y) & \text{ if } X,Y \in \D,         \\
            \{*\}          & \text{ if } X \in \C, Y \in \D, \\
            \emptyset      & \text{ if } X \in \D, Y\in \C,
        \end{cases}
    \]
    with composition defined such that $\C \hookrightarrow \C \star \D \hookleftarrow \D$ are functors.
\end{definition}
\begin{remark}
    These inclusions are isomorphisms to full subcategories of the join.
    It is usual to abuse notation a bit and identify $\C$ and $\D$ with these subcategories.
\end{remark}
\begin{example}
    Maybe the most important examples of joins, at least in this text, are the left and right cone of a category.
    Letting $[0]$ denote the category with one object and one morphism, we denote by $\C^{\triangleleft}: = [0] \star \C$ the right cone of a $1$-category $\C$ and $\C^{\triangleright}:=\C \star [0]$ the left cone of $\C$.
    In practice, the right cone of $\C$ is the category obtained by adjoining an additional object $X_0$ to $\C$ and for every $C \in \C^{\triangleright}$ a unique morphism $X_0 \to C$ so that $X_0$ becomes terminal in $\C^{\triangleright}$.
    Dually, the left cone is obtained by adjoining an additional object which becomes initial in $\C^{\triangleleft}$.
    \newline
    The usefulness of cones materializes when considering limits and colimits.
    %For a functor $F: \C \to \D$, a colimit of $F$ is a functor $\hat{F}$ initial among functors extending $F$ to $\C^{\triangleright}$ and a limit of $F$ is a functor terminal among functors 
    Lurie denotes the category of functors extending $F$ to the cones by $\Fun_F(\C^{\triangleright}, \D) : = \{G \in \Fun(\C^{\triangleright}, \D) | G|_{\C} = F\}$ and $\Fun_F(\C^{\triangleleft}, \D): = \{G \in \Fun(\C^{\triangleleft}, \D) | G|_{\C} = F\}$ and colimits and limits of $F$ can be identified with initial and terminal objects in $\Fun_F(\C^{\triangleright}, \D)$ and $\Fun_F(\C^{\triangleleft}, \D)$ respectively.
\end{example}
We now define the join of two simplicial sets.
\begin{definition}
    Let $X, Y \in \sset$.
    We define the join $X\star Y$ on $n$-cells:
    \[
        (X\star Y)_n : = \coprod_{[n] = [n_1] \sqcup [n_2]} X_{n_1} \times X_{n_2},
    \]
    where $[n_1], [n_2] \in \Delta \cup \emptyset : = \Delta_+$ and $\sqcup: \Delta_+ \times \Delta_+ \to \Delta_+$ is the ordered disjoint union.
    That is, $[p] \sqcup [q] = [p+1+q]$.
    We consider $X_{-1} = * = Y_{-1}$.
\end{definition}
\end{document}
