\documentclass[../../thesis.tex]{subfiles}

\begin{document}
\section{Stable $\infty$-categories}
Quoting Lurie \cite{HA}, ``The theory of stable $\infty$-categories an be regarded as an axiomatization of the essential features of stable homotopy theory: most notably, that fiber sequences and cofiber sequences are the same.''
For the purpose of this thesis (proving Verider Duality), we really don't need much more of the theory than the fact that pullbacks squares are always also pushout squares, but nevertheless we think it useful to state the basic theory and examples of the categories we care about.
\begin{definition}[{\cite[Definition 1.1.1.1.]{HA}}]
    A zero-object, usually denoted $0$, in an $\infty$-category is an object that is both initial and terminal.
    A pointed $\infty$-category is an $\infty$-category containing a zero-object.
\end{definition}
\begin{remark}
    An object $0\in \C$ is a zero-object if $\map_{\C}(X,0)$ and $\map_{\C}(0,X)$ are contractible for every object $X\in \C$ and such an object is determined up to equivalence (we saw that initial and terminal objects are unique up to a contractible space).
\end{remark}
\begin{remark}[{\cite{HA}[Remark 1.1.1.3.]}]
    Let $\C$ be a pointed $\infty$-category.
    For any $X,Y \in \C$, the natural morphism
    \[
        \map_{\C}(X, 0) \to \map_{\C}(0, Y) \to \map_{\C}(X,Y)
    \]
    has contractible domain, which means that we have a well-defined zero-morphism $X\to Y$ in $\h\C$.
\end{remark}
\begin{definition}[{\cite{HA}[Definition 1.1.1.4.]}]
    Let $\C$ be a pointed $\infty$-category.
    We define a triangle in $\C$ to be a diagram $\Delta^1 \times \Delta^1 \to \C$, depicted as
    \begin{equation}\label{triangle}\begin{tikzcd}
            X & Y \\
            0 & Z
            \arrow["f", from=1-1, to=1-2]
            \arrow[from=2-1, to=2-2]
            \arrow["g", from=1-2, to=2-2]
            \arrow[from=1-1, to=2-1]
        \end{tikzcd}\end{equation}
    where $0$ is a zero-object of $\C$.
\end{definition}
\begin{definition}
    We say that a triangle is a fiber sequence if it is a pullback square and a cofiber sequence if it is a pushout square.
\end{definition}
\begin{definition}[{\cite{HA}[Definition 1.1.1.6.]}]
    For a morphism $g: Y \to Z$ in a pointed $\infty$-category $\C$, we say that a fiber of $g$ is a fiber seqeunce as depicted in \ref{triangle}.
    For a morphism $f: X \to Y$ in a pointed $\infty$-category $\C$, we say that a cofiber of $f$ is a cofiber seqeunce as depicted in \ref{triangle}.
    We will abuse terminology by writing $X = \fib(g)$ and $Z=\cofib(f)$.
\end{definition}
\begin{remark}[{\cite{HA}[Remark 1.1.1.8.]}]
    The functor $\cofib: \Fun(\Delta^1, \C)\to \C$ can be identified with a left adjoint to the left Kan extension functor $\C \simeq \Fun(\{1\}, \C) \to \Fun(\Delta^1, \C)$ which associates a zero morphism $0 \to X$ to each object $X \in \C$.
    This means that $\cofib$ preserves all colimits in $\Fun(\Delta^1, \C)$.
\end{remark}
We now define stable $\infty$-categories:
\begin{definition}[{\cite{HA}[Definition 1.1.1.9.]}]
    We say that an $\infty$-category is stable if it satisfies the following conditions:
    \begin{enumerate}
        \item It is pointed.
        \item Any morphism in $\C$ has a fiber and a cofiber.
        \item A triangle in $\C$ is a fiber sequence if and only if it is a cofiber sequence.
    \end{enumerate}
\end{definition}
\end{document}