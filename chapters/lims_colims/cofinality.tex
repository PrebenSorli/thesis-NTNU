\documentclass[../../thesis.tex]{subfiles}

\begin{document}
\section{Cofinality}
\TODO{Write something intuitive about why we care about cofinality and some historical perspectives.}
\begin{definition}{\cite[Definition 4.1.1.1]{HTT}}
    Let \(p:S\to T \in \sset \). We say $p$ is \emph{cofinal} if, for any right fibration $X\to T$, the induced map of simplicial sets
    \[
        \MAP_T(T,X) \to \MAP_T(S,X)
    \]
    is a homotopy equivalence.
\end{definition}
We will probably want to write more about cofinality later, but for now we state a result that will be extremely helpful later.
\begin{theorem}{\cite[\href{https://kerodon.net/tag/02NY}{Theorem 02NY}]{kerodon}\label{superlemma}}
    Let \(f: \C \to \D\) be a map of simplicial sets, where $\D$ is an $\infty$-category.
    Then the following conditions are equivalent:
    \begin{enumerate}
        \item The functor $f$ is left cofinal,
        \item For every $D\in \D$, the simplicial set $\C\times_\D \D_{/D}$ is weakly contractible;
    \end{enumerate}
    and the following conditions are equivalent:
    \begin{enumerate}
        \item The functor $f$ is right cofinal,
        \item For every $D\in \D$, the simplicial set $\C\times_\D \D_{D/}$ is weakly contractible.
    \end{enumerate}
\end{theorem}
\end{document}