\documentclass[../thesis.tex]{subfiles}

\begin{document}
%\begin{abstract}
We prove Lurie's $\infty$-categorical version of Verdier duality.
To this end, we introduce $\infty$-categories and some important results and constructions.
Specifically, we introduce limits and colimits and state some results that greatly aid our calculations.
Before introducing $\infty$-categorical versions of Kan extensions, we briefly recollect the definitions and main results of Kan extensions in $1$-categories.
We give a short account of the history behind sheaf theory before showing an equivalence between sheaves and $\K$-sheaves of locally compact Hausdorff spaces valued in $\infty$-categories.
Finally, we recall briefly the classical notion of Verdier duality before using all the above theory to prove the $\infty$-categorical version.


    {\let\clearpage\relax\chapter*{Sammendrag}}
Vi beviser Luries $\infty$-kategoriske versjon av Verdier-dualitet.
For å oppnå dette introduserer vi $\infty$-kategorier og noen viktige resultat og konstruksjoner.
Mer spesifikt, så introduserer vi grenser og kogrenser og presenterer noen resultater som er svært nyttige for å gjøre utregninger.
Før vi introduserer $\infty$-kategoriske versjoner av kanutvidelser, gjentar vi kort definisjonene og de viktigste resultatene for kanutvidelser i $1$-kategorier.
Vi gir en kort gjennomgang av historien bak knippeteori før vi viser en ekvivalens mellom knipper og $\K$-knipper på lokalt kompakte Hausdorff topologiske rom med verdier i $\infty$-kategorier.
Til slutt presenterer vi klassisk Verdier-dualitet før vi bruker all teorien nevnt over til å bevise den $\infty$-kategoriske varianten.

% In 1965, Jean-Louis Verdier introduced Verdier duality for locally compact topological spaces, thus generalizing the classical theory of Poincaré duality for manifolds.
% Verdier duality is a cohomological duality that allows exchanging cohomology for cohomology with compact support.
% More precisely it states that the derived functor of the compactly supported direct image functor has a right adjoint in the derived category of sheaves.
% By using sheaf cohomology one can derive the classical Poincaré duality as a special case.
% In his book ``Higher Algebra'' Jacob Lurie extends the theory to the $\infty$-categorical setting by showing there is an equivalence between sheaves and cosheaves valued in stable $\infty$-categories.
% This thesis follows this proof closely, expanding and adding details where necesarry.
% To introduce the relevant background on sheaves and $\K$-sheaves valued in stable $\infty$-categories we introduce and utilize Kan extensions, a ubiquitous concept in category theory, as well as giving an expository account of the basic theory of $\infty$-categories.



%\end{abstract}
\end{document}