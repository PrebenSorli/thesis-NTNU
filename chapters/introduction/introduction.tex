\documentclass[../../thesis.tex]{subfiles}
\begin{document}
In 1965, Jean-Louis Verdier introduced Verdier duality for locally compact topological spaces \cite{Verdier95}, thus generalizing the classical theory of Poincaré duality for manifolds.
A few years earlier, Grothendieck had introduced a version in algebraic geometry by proving a duality in étale cohomology for schemes.
Classical Verdier duality was introduced as a topological analog of this duality.
It is a cohomological duality that allows exchanging cohomology for cohomology with compact support.
More precisely, it states that the derived functor of the compactly supported direct image functor has a right adjoint in the derived category of sheaves.
By using sheaf cohomology, one can derive the classical Poincaré duality as a special case.
In his book ``Higher Algebra'' \cite{HA}, Jacob Lurie extends the theory to the $\infty$-categorical setting by showing there is an equivalence between sheaves and cosheaves valued in stable $\infty$-categories.
The main goal of the thesis is therefore to prove the following theorem:
\begin{theorem*}[\ref{VerdierDuality}]
    Let $X$ be a locally compact Hausdorff space and $\C$ be a stable $\infty$-category with small limits and colimits.
    Then we have an equivalence of $\infty$-categories
    \[
        \mathbb{D} \colon \Shv(X;\C) \simeq \mathrm{CoShv}(X;\C).
    \]
\end{theorem*}
Instead of introducing a separate theory of cosheaves, we will simply make use of the idenfication $\mathrm{CoShv}(X;\C)\simeq \Shv(X;\C^{op})^{op}$ and more or less take that as our definition of the category of cosheaves.
As is the case for classical Verdier duality, Lurie's theorem has multiple interesting consequences.
Lurie himself uses it to prove a non-abelian version of Poincaré duality in \cite{HA}.
Additionally, Volpe shows in \cite{Volpe} that Verdier duality can be used to construct a six functor formalism for sheaves with values in symmetric monoidal $\infty$-categories which are stable and bicomplete.


Following Lurie, we prove the equivalence on the level of $\K$-sheaves.
Lurie introduced $\K$-sheaves in his book ``Higher Topos Theory'' \cite{HTT} and used it to show the second big theorem in this thesis:
\begin{theorem*}[\ref{7.3.4.9}]
    There is an equivalence of $\infty$-categories
    \[
        \Shv(X;\C) \simeq \Shv_{\K}(X;\C).
    \]
\end{theorem*}
We will follow Lurie's proof of these two theorems closely, filling in details and calculations where we deem necessary.
To introduce the relevant background on sheaves and $\K$-sheaves valued in stable $\infty$-categories we introduce and utilize Kan extensions, a ubiquitous concept in category theory, as well as giving an expository account of the basic theory of $\infty$-categories.
This expository account can safely be skipped for the reader to whom $\infty$-categories are well-known.
\end{document}